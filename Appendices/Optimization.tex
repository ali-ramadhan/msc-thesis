% If problems with the headers: get headings in appendix etc. right
%\markboth{\spacedlowsmallcaps{Appendix}}{\spacedlowsmallcaps{Appendix}}

\chapter{Mathematical Optimization}

\section{Linear optimization}
\section{Convex optimization}
\section{Nonlinear optimization}

% If the objective function f is linear and the constrained space is a polytope, the problem is a linear programming problem, which may be solved using well-known linear programming techniques such as the simplex method.
% If the objective function is concave (maximization problem), or convex (minimization problem) and the constraint set is convex, then the program is called convex and general methods from convex optimization can be used in most cases.
% If the objective function is quadratic and the constraints are linear, quadratic programming techniques are used.
% If the objective function is a ratio of a concave and a convex function (in the maximization case) and the constraints are convex, then the problem can be transformed to a convex optimization problem using fractional programming techniques.

Lorem ipsum at nusquam appellantur his, ut eos erant homero
concludaturque. Albucius appellantur deterruisset id eam, vivendum
partiendo dissentiet ei ius. Vis melius facilisis ea, sea id convenire
referrentur, takimata adolescens ex duo. Ei harum argumentum per. Eam
vidit exerci appetere ad, ut vel zzril intellegam interpretaris.
\graffito{More dummy text.}

Theorems can easily be defined

\begin{definition}
  Let $f$ be a function whose derivative exists in every point, then $f$ is 
  a continuous function.
\end{definition}

\begin{theorem}[Pythagorean theorem]
  \label{pythagorean}
  This is a theorema about right triangles and can be summarised in the next 
  equation 
  \[ x^2 + y^2 = z^2 \]
\end{theorem}

\begin{proof}
  The theorem can be proved algebraically using four copies of a right triangle with sides $a$, $b$, and $c$, arranged inside a square with side $c$ as in the top half of the diagram. The triangles are similar with area $\tfrac12ab$, while the small square has side $b-a$ and area $(b-a)^2$. The area of the large square is therefore
  $$(b-a)^2+4\frac{ab}{2} = (b-a)^2+2ab = a^2+b^2$$  
  But this is a square with side $c$ and area $c^2$, so $c^2 = a^2 + b^2$.
\end{proof}

And a consequence of theorem \ref{pythagorean} is the statement in the next 
corollary.