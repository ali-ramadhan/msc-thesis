% If problems with the headers: get headings in appendix etc. right
%\markboth{\spacedlowsmallcaps{Appendix}}{\spacedlowsmallcaps{Appendix}}

\chapter{Bayesian Statistics}\label{ch:stats}

\section{Fundamentals of probability theory}

\section{Bayesian inference}

\section{Markov chain Monte Carlo}

\section{Inverse problems}
% Inverse problems, ill-posed and well-posed problems, Tikhonov regularization, ...? (FROG phase retrieval might be a good example of finding an arbitrary distribution -- inverse problem?)

Lorem ipsum at nusquam appellantur his, ut eos erant homero
concludaturque. Albucius appellantur deterruisset id eam, vivendum
partiendo dissentiet ei ius. Vis melius facilisis ea, sea id convenire
referrentur, takimata adolescens ex duo. Ei harum argumentum per. Eam
vidit exerci appetere ad, ut vel zzril intellegam interpretaris.
\graffito{More dummy text.}

Theorems can easily be defined

\begin{definition}
  Let $f$ be a function whose derivative exists in every point, then $f$ is 
  a continuous function.
\end{definition}

\begin{theorem}[Pythagorean theorem]
  \label{pythagorean}
  This is a theorema about right triangles and can be summarised in the next 
  equation 
  \[ x^2 + y^2 = z^2 \]
\end{theorem}

\begin{proof}
  The theorem can be proved algebraically using four copies of a right triangle with sides $a$, $b$, and $c$, arranged inside a square with side $c$ as in the top half of the diagram. The triangles are similar with area $\tfrac12ab$, while the small square has side $b-a$ and area $(b-a)^2$. The area of the large square is therefore
  $$(b-a)^2+4\frac{ab}{2} = (b-a)^2+2ab = a^2+b^2$$  
  But this is a square with side $c$ and area $c^2$, so $c^2 = a^2 + b^2$.
\end{proof}

And a consequence of theorem \ref{pythagorean} is the statement in the next 
corollary.