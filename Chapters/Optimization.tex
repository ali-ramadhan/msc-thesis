\chapter{Geometry reconstruction as an optimization problem}\label{ch:optimization}

Much progress was made over the lookup table by treating the geometry reconstruction problem as a constrained nonlinear convex optimization such that MATLAB's \texttt{fmincon} function an be relied on. It relies on trust regions and uses an interior-point algorithm. This worked especially well in the case of triatomic molecules however four-atom systems proved incredibly difficult to tackle here. This was due to the exponential increase in the number of saddle points with dimensionality \footnotemark making the problem highly non-convex and unsuitable for \texttt{fmincon}.

\footnotetext{Recall that triatomic molecules have three degrees of freedom resulting in a problem of dimension 3 while four-atom systems have six. That is, $3N+6$ for an $N$-atom system.}

\section{Optimization problems}
\section{Previous work employing a simplex algorithm}
\index{Simplex algorithm}
\index{Geometry reconstruction!Simplex algorithm}
\citet{Brichta09} proposed the reconstruction of small triatomic molecules using a simplex algorithm. Unfortunately they only report on the reconstruction of molecular structures based on simulated data for carbon dioxide and formaldehyde. I could not use this algorithm to find the geometries of CO2 or OCS from real data.

\section{Current implementation}

\section{Reconstructions of carbon dioxide and carbonyl sulfide}

\section{Advantages, disadvantages and problems}

\section{Lessons learnt}