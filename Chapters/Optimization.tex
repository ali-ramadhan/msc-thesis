\chapter{Geometry reconstruction using constrained nonlinear optimization}\label{ch:optimization}

Much progress was made over the lookup table by treating the geometry reconstruction problem as a constrained nonlinear convex optimization such that MATLAB's \texttt{fmincon} function an be relied on. It relies on trust regions and uses an interior-point algorithm. This worked especially well in the case of triatomic molecules however four-atom systems proved incredibly difficult to tackle here. This was due to the exponential increase in the number of saddle points with dimensionality \footnotemark making the problem highly non-convex and unsuitable for \texttt{fmincon}.

\footnotetext{Recall that triatomic molecules have three degrees of freedom resulting in a problem of dimension 3 while four-atom systems have six. That is, $3N+6$ for an $N$-atom system.}

\section{Optimization problems}
\section{Previous work employing a simplex algorithm}
\index{Simplex algorithm}
\index{Geometry reconstruction!Simplex algorithm}
\citet{Brichta09} proposed the reconstruction of small triatomic molecules using a simplex algorithm. It should not be confused with the more famous simplex algorithm, also an optimization algorithm but for linear programming, taught in almost every introductory optimization course. The algorithm employed should really should be referred to as the Nelder-Mead method, downhill simplex method, or amoeba method to avoid confusion between the two.\footnotemark

\footnotetext{We refer to it as the Nelder-Mead method for this chapter in concordance with Wikipedia.}

The Nelder-Mead algorithm is an ad-hoc or heuristic algorithm for nonlinear optimization that can be used without computing derivatives of the objective function\footnotemark. It was first generalized to minimizing functions by \citet{Nelder65} based off ideas by \citet{Spendley62}. It has enjoyed widespread popularity due to its ease of implementation and intuitive inner workings but it is not appropriate to every problem. In fact, it is not guaranteed to converge and thus fails when applied to some problems. It can even converge to non-stationary points in some cases \citep{McKinnon98}. Later publications would sometimes introduce tweaks to the Nelder-Mead method that would improve its performance on a specific problem. Unfortunately, I believe geometry reconstruction is not an appropriate problem for the Nelder-Mead method.

\footnotetext{\citet{Wright10} provides a great discussion of the Nelder-Mead method, ending with a comment by John Nelder regarding his algorithm, ``Mathematicians hate it because you can’t prove convergence; engineers seem to love it because it often works.''}

Unfortunately they only report on the reconstruction of molecular structures based on simulated data for carbon dioxide and formaldehyde. I could not use this algorithm to find the geometries of CO2 or OCS from real data.

\section{Current implementation}

\section{Geometry reconstructions done by lookup table}
\subsection{A symmetric triatomic molecule: carbon dioxide}
\subsection{An asymmetric triatomic molecule: carbonyl sulfide}
\subsection{A four-atom system: acetylene}

\section{Lessons learnt}
\subsection{The extreme importance of covariances and joint distributions}
\subsection{Geometry reconstruction is highly sensitive to uncertainty in the momentum vectors}

\section{Future directions}