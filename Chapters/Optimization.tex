\chapter{Geometry reconstruction using constrained nonlinear optimization}\label{ch:optimization}

\vspace{-1.5 em}
\begin{addmargin}[-0.5cm]{0cm}
  \minitoc
\end{addmargin}
\hrule
\vspace{1.5 em}

Much progress was made over the lookup table by treating the geometry reconstruction problem as a constrained nonlinear convex optimization such that MATLAB's \texttt{fmincon} function an be relied on. It relies on trust regions and uses an interior-point algorithm. This worked especially well in the case of triatomic molecules however four-atom systems proved incredibly difficult to tackle here. This was due to the exponential increase in the number of saddle points with dimensionality \footnotemark making the problem highly non-convex and unsuitable for \texttt{fmincon}.

\footnotetext{Recall that triatomic molecules have three degrees of freedom resulting in a problem of dimension 3 while four-atom systems have six. That is, $3N+6$ for an $N$-atom system.}

\section{An aside on optimization problems}

\section{Current implementation}

\section{Geometry reconstructions}

\section{Conclusions and lessons learnt}
\subsection{The extreme importance of covariances and joint distributions}
% \subsection{Geometry reconstruction is highly sensitive to uncertainty in the momentum vectors}

% \section{Future directions}
% Ref curse of dimensionality, maybe look to other optimization algorithms, e.g. the high dimensionality saddle point avoiding one.