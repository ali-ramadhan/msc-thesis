\chapter{Geometry reconstruction using a lookup table}\label{ch:lookupTable}

We begin our attempts at reconstructing geometries by taking a very simple approach.

\section{Lookup tables: what are they good for?}
As the name would suggest, a lookup table catalogues a relationship between two sets such that anyone wishing to obtain the mapping between the sets may simply find the object of interest. Lookup tables tend to be more useful in one direction \citep{Maher01}.

The most familiar lookup table may be the multiplication times table that every elementary school student is familiar with where sets of two integers are mapped to their product \citep{Hayashi97}. 

% Implementation (+ extra convergence cube)
\section{Implementation to reconstruct triatomic molecules}
\index{Lookup table}
\index{Geometry reconstruction!Lookup table}
In this approach, many Coulomb explosions are simulated for a wide variety of structures, and the resulting momentum vectors from each simulation are stored. Thus you have a mapping from molecular structures to momentum vectors. To determine the structure belonging to a certain set of observed momentum vectors, you simply read the table in reverse. This approach is simple to implement, very quick by design, and front-loads the computation which may be desirable for large data sets. However, of course, it has an exponential time and space complexity $\mathcal{O}(e^{3N-6})$ where $N$ is the number of atoms.

\section{Reconstructions of carbon dioxide and carbonyl sulfide}

\section{Computational complexity}

\section{Advantages, disadvantages and problems}

\section{Lessons learnt}