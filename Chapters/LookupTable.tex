\chapter{Geometry reconstruction by lookup table}\label{ch:lookupTable}

\vspace{-1.5 em}
\begin{addmargin}[-0.5cm]{0cm}
  \minitoc
\end{addmargin}
\hrule
\vspace{1.5 em}

\noindent
We begin our attempts at reconstructing geometries by taking a very simple approach, by using a lookup table. Simulating Coulomb explosions is computationally cheap, so we can simulate the explosion of many geometries and create a large map of geometries to asymptotic momentum vectors. Then reconstructing a geometry is simply a matter of looking up the geometry that produces the closest momentum vector arrangement. In this chapter we will describe an implementation of this approach and report on the geometries reconstructed using it. This will lead us to some new insights regarding the geometry reconstruction problem in general, motivating further investigation and possible enhancements to the method. Finally we will investigate its drawbacks, which further motivate the need for a more advanced approach and lay the foundations for the next chapter (see chapter \ref{ch:optimization}). Before fully abandoning this approach, we will also discuss its usefulness in future applications.

First however, we will take a brief look at lookup tables and motivate the need for it by investigating the failures of a previous attempt employing the Nelder-Mead simplex method.

\section{An aside on lookup tables}
The idea of using a lookup table to speed up calculations is as old as mathematics itself with examples dating back to some of the earliest mathematical texts produced by ancient Egyptian scribes during the Twelfth Dynasty of Egypt (circa 1990--1800 BC) \citep[p. 1, footnote 4]{Neugebauer45}. The Egyptian Mathematical Leather Roll contains perhaps the first such complete table and tabulates 26 sums of unit fractions equaling another unit fraction. \citet{Glanville27} reports on the leather roll,\footnotemark~ housed at the British Museum, and gives a photograph (figure \ref{fig:EMLR}) and schematic (figure \ref{fig:EMLRschematic}) of the table.

\footnotetext{The Egyptian Mathematical Leather Roll (and the more famous Rhine Mathematical Papyrus) was brought to the British Museum in 1864, however it took decades before archaelogists knew how to treat the leather to prevent its disintegration upon unrolling it.}

An even more impressive lookup table is found in the extensive Rhind Mathematical Papyrus, which (seemingly) methodologically expresses the fractions $2/n$ for odd $n \in \lbrace 3, 5, \dots, 101 \rbrace$ as the sum of 2--4 unit fractions! \citet{Gillings82} reports extensively on this table and the several dozen problems posed and solved on the papyrus, making extensive use of the table. The same table, verbatim, has found to be in use by scribes more than a millenium after its creation suggesting that it was of great utility. To this day, scholars argue about exactly how the scribes knew to construct the table and what methods they used \citep{Gillings74,Abdulaziz08}. \citet{Neugebauer45} reports on mathematical cuneiform texts used by the ancient Babylonians during similar time periods.\footnotemark~ Apparently, we are far from being the first to utilize a lookup table to speed up calculations.

\footnotetext{Other historical lookup tables of importance include a 98-column Roman multiplication table from 493 A.D. \citep{Maher01} and an ancient Indian sine table circa 499 C.E. \citep{Hayashi97}.}

% The most familiar lookup table may be the multiplication times table that every elementary school student is familiar where sets of two integers are mapped to their product. Lookup tables have been employed since antiquity and one of the earliest surviving examples is a 98-column multiplication table from 493 A.D. attributed to the Roman, Victorius of Aquitaine \citep{Maher01}. Before the advent of the calculator and for much of scientific history, extensive logarithm tables were used to look up their values to several decimal places and to speed up computations. The earliest such tables date back to the ancient greeks although they have since been lost, and the earliest surviving one is a sine table by the ancient Indian mathematician \={A}ryabha\d{t}a circa 499 C.E. \citep{Hayashi97}.

\begin{figure}
  \centering
  \includegraphics[width=\textwidth]{gfx/EMLR}
  \caption[Photograph of columns 3 and 4 of the Egyptian Mathematical Leather Roll.]
  {Photograph of columns 3 and 4 of the Egyptian Mathematical Leather Roll. Columns 3 and 4 are duplicates of columns 1 and 2. For example, row 9 of column 3 translates to $\displaystyle \frac{1}{50}\frac{1}{30}\frac{1}{150}\frac{1}{400} \quad \mathrm{it \; is} \quad \frac{1}{16}$ in modern notation where addition is implied. Figure \ref{fig:EMLRschematic} is a schematic of the table photographed here. Figure from \citet{Glanville27} which is accompanied by a translation.}
  \label{fig:EMLR}
\end{figure}

\begin{figure}
  \centering
  \includegraphics[width=\textwidth]{gfx/EMLRschematic}
  \caption[Schematic of columns 1 and 2 of the Egyptian Mathematical Leather Roll.]
  {Schematic of columns 1 and 2 of the Egyptian Mathematical Leather Roll. Columns 3 and 4 are duplicates of columns 1 and 2. Figure \ref{fig:EMLR} is a photograph of the table outlined here. Figure from \citet{Glanville27} which is accompanied by a translation.}
  \label{fig:EMLRschematic}
\end{figure}

% Of course since then lookup tables have found an enormous number of uses in computer science. Arrays are ubiquitous objects in procedural programming languages and so is the more general dictionary object, especially in Python. Sine tables are still stored in calculators for quick trigonometric computations using the CORDIC algorithm and 3D lookup tables are used in image processing to store colormaps. In each case the speed offered by a lookup table once it has been generated is the main reason for their use.

\section{Previous attempt using the Nelder-Mead method}
\index{Simplex algorithm}
\index{Geometry reconstruction!Simplex algorithm}
\citet{Brichta09} proposed the reconstruction of small triatomic molecules using a simplex algorithm. It should not be confused with the more famous simplex algorithm, also an optimization algorithm but for linear programming, taught in almost every introductory optimization course. The algorithm employed should really should be referred to as the Nelder-Mead method, downhill simplex method, or amoeba method to avoid confusion between the two.\footnotemark

\footnotetext{We refer to it as the Nelder-Mead method for this chapter in alignment with Wikipedia.}

The Nelder-Mead algorithm is an ad-hoc or heuristic algorithm for nonlinear optimization that can be used without computing derivatives of the objective function\footnotemark. It was first generalized to minimizing functions by \citet{Nelder65} based off ideas by \citet{Spendley62}. It has enjoyed widespread popularity due to its ease of implementation and intuitive inner workings but it is not appropriate to every problem. In fact, it is not guaranteed to converge and thus fails when applied to some problems. It can even converge to non-stationary points in some cases \citep{McKinnon98}. Later publications would sometimes introduce tweaks to the Nelder-Mead method that would improve its performance on a specific problem. Unfortunately, I believe geometry reconstruction is not an appropriate problem for the Nelder-Mead method (see section \ref{ssec:simplexFail}).

\footnotetext{\citet{Wright10} provides a great discussion of the Nelder-Mead method, ending with a comment by John Nelder regarding his algorithm, ``Mathematicians hate it because you can’t prove convergence; engineers seem to love it because it often works.'' \citet[section 10.4]{NumericalRecipes} describes the algorithm in detail and provides a well-commented C++ implementation.}

\subsection{Previous reconstructions}
Unfortunately, \citet{Brichta09} only report on the reconstruction of molecular structures based on a few simulated geometries for carbon dioxide (\ch{CO2}) and formaldehyde (\ch{CH2O}). \citet{Brichta07} claims to have used this algorithm to report reconstructions of real \ch{CO2} geometries a couple of years earlier (see figure \ref{fig:simplexJPhysB}), which seems ripe for discussion and investigation, however the work is not even discussed. It was also used by \citet{Bocharova11} to report on the molecular structure of \ch{CO2} $(2,2,2)$ (see figure \ref{fig:simplexPRL}) however they do not report the set of geometries used to form the initial simplex. While they both produce a nice and intuitive plots, showing what appears to be an approximation to a molecular wavefunction with two broad position distributions, we will later see that this is deceptive and may not even represent physical geometries. 

\begin{figure}
  \centering
  \includegraphics[width=\textwidth]{gfx/SimplexJPhysB}
  \caption[Reconstructed \ch{CO2} geometries using the Nelder-Mead method.]
  {Reconstructed \ch{CO2} geometries after Coulomb explosion by \SI{50}{\fs} laser pulses using the Nelder-Mead method for the (a) $(1,1,1)$, (b) $(1,2,1)$, (c) $(1,1,2)$, (d) $(1,2,2)$, (e) $(2,1,2)$, and (f) $(2,2,2)$ fragmentation channels. The carbon atom is placed at the origin with the higher charged oxygen atom in the +$x$-quadrant. The three circles represent the equilibrium or ground state geometry of \ch{CO2}, which is plotted as perfectly linear even though it should have a bond angle of \SI{172.5}{\degree} \citep{Siegmann02,Mathur92}. Each plot is reported to contain approximately $10^3$ events (or geometries), however it is hard to believe that (a) and (d) contain similar amounts of data. Figure from \citet{Brichta07}.}
  \label{fig:simplexJPhysB}
\end{figure}

\begin{SCfigure}
  \centering
  \includegraphics[width=0.5\textwidth]{gfx/SimplexPRL}
  \caption[Reconstructed \ch{CO2} in the $(2,2,2)$ charge state using the Nelder-Mead method.]
  {Reconstructed \ch{CO2} geometries in the $(2,2,2)$ charge state after explosion by a \SI{7}{\fs} laser pulse using the Nelder-Mead method. Although unreported, the triangles presumably pinpoint the centroid (or average) position of each atom while the the 
    
    They report a molecular geometry of $\langle r_\mathrm{CO} \rangle \approx \SI{1.3}{\angstrom}$, $\langle \theta_\mathrm{OCO} \rangle \approx 168\degree$ which is close the equilibrium geometry of $\langle r_\mathrm{CO} \rangle \approx \SI{1.16}{\angstrom}$ \citep{ChemistryOfTheElements} and $\langle \theta_\mathrm{OCO} \rangle \approx 172\degree$ \citep{Siegmann02,Mathur92}. Figure from \citet{Bocharova11}.}
  \label{fig:simplexPRL}
\end{SCfigure}

\subsection{Inability to reliably reconstruct triatomic molecules} \label{ssec:simplexFail}

In attempting to apply the Nelder-Mead method to a the problem of geometry reconstruction, we found the results to vary greatly depending on the set of geometries chosen to form the initial simplex. To investigate this worrying aspect of the algorithm, we decided to see whether it could reconstruct simulated geometries. That is, we would chose a molecular geometry and use it to simulate a Coulomb explosion (see section \ref{sec:simulating}) and compute the resulting asymptotic momentum vectors. These momentum vectors are then fed into the Nelder-Mead algorithm to see whether it could recover the molecular geometry we started off with.

Simulated geometries are used for this analysis mainly because we can check exactly how far off the algorithm is from the correct solution. The simulated geometry only experiences the Coulomb force and so a solution is guaranteed by our classical simulation as no quantum mechanical effects occur. Simulated geometries are thus the easiest geometries to reconstruct and serious doubts regarding an algorithm's utility must be raised if it cannot reconstruct them. We do not know the geometries \textit{a priori} when reconstructing real geometries and so we need a trustworthy reconstruction algorithm in order to trust any geometries it produces.

What we found confirmed the unreliable nature of the Nelder-Mead method for reconstructing molecular geometries. We attempted to reconstruct molecular structures for \ch{CO2} and \ch{OCS}, both in the $(2,2,2)$ fragmentation channel. Starting from their equilibrium geometries we varied one parameter at a time to test whether the Nelder-Mead method could recover the geometry. Figure \ref{fig:CO2SimplexCalibrationPlots} shows the reconstruction results for \ch{CO2} $(2,2,2)$ and figure \ref{fig:OCSSimplexCalibrationPlots} shows the same for \ch{OCS} $(2,2,2)$.

\begin{figure}
  \centering
  \includegraphics[width=\textwidth]{Plots/CO2SimplexCalibrationPlots}
  \caption[Testing the Nelder-Mead method's ability to reconstruct \ch{CO2} (2,2,2) geometries]
  {Testing the Nelder-Mead method's ability to reconstruct \ch{CO2} (2,2,2) geometries by starting with the ground-state geometry of \ch{CO2} and varying each parameter one-by-one. In the first row, the first \ch{C-O} bond length ($r_{12}$) was varied to create an ``input geometry'' which underwent a simulated Coulomb explosion. The resulting momentum vectors from the explosion were fed into the Nelder-Mead algorithm which converged to a geomety, the ``output geometry''. The second \ch{C-O} bond length ($r_{23}$) and the bond angle $\theta$ were varied in the second and third rows respectively. Solid black lines indicate the expected output geometry, so deviations indicate a failure on the algorithm's ability to reconstruct the geometry. We see that the algorithm can accurately reconstruct the geometry when a single bond length is varied but completely fails once the bond angle is below approximately $176\degree$, thus simply producing zeros.}
  \label{fig:CO2SimplexCalibrationPlots}
\end{figure}

\begin{figure}
  \centering
  \includegraphics[width=\textwidth]{Plots/OCSSimplexCalibrationPlots}
  \caption[Testing the Nelder-Mead method's ability to reconstruct \ch{OCS} (2,2,2) geometries.]
  {Testing the Nelder-Mead method's ability to reconstruct \ch{OCS} (2,2,2) geometries by starting with the ground-state geometry of \ch{CO2} and varying each parameter one-by-one. In the first row, the \ch{C-O} bond length ($r_\textrm{CO}\equiv r_{12}$) was varied to create an ``input geometry'' which underwent a simulated Coulomb explosion. The resulting momentum vectors from the explosion were fed into the Nelder-Mead algorithm which converged to a geomety, the ``output geometry''. The \ch{C-S} bond length ($r_\textrm{CS}\equiv r_{23}$) and the bond angle $\theta$ were varied in the second and third rows respectively. Solid black lines indicate the expected output geometry, so deviations indicate a failure on the algorithm's ability to reconstruct the geometry. We see that the algorithm can accurately reconstruct the geometry when a single bond length is varied but performs worse and worse as the molecule bends, even slightly.}
  \label{fig:OCSSimplexCalibrationPlots}
\end{figure}

The Nelder-Mead method successfully retrives the bond lengths for the majority of cases but not the bond angles. In the case of \ch{CO2} it seems to completely fail at retreiving the geometry once the bond angle falls below approximately $176\degree$. For \ch{OCS}, retrieved geometries get worse and worse as the molecule bends, even slightly.

A more thorough analysis should vary multiple parameters at once to more fully explore the state space of the problem, however if the algorithm cannot even reconstruct geometries that different from the equilibrium state by a single parameter as we have done, then the results of a more thorough analysis will probably be even more worrying. We already have enough information to distrust the Nelder-Mead method and begin searching for a new approach.

It should be noted that the Nelder-Mead algorithm, especially in our case, is quite sensitive to the geometries chosen to represent the initial simplex. Changing them could significantly impact the algorithm's ability to converge on the correct geometry. Of course, this does suggest that there may exist a set of initial geometries that significantly improve the algorithm's reliability, however, after many attempts I could not find such a set. Some choices improved the retreival of bond angles at the cost of diminished retreival of the bond lengths. The initial simplex used to produce figures \ref{fig:CO2SimplexCalibrationPlots} and \ref{fig:OCSSimplexCalibrationPlots} was chosen to maximize the number of successful reconstructions for very straight molecules which constitute the majority of cases, however molecules with bond angles $\theta < 175\degree$ are very common and so even with this choice, we are very unsure about the accuracy of a majority of reconstructions.

% TODO: Include my initial simplex.

Perhaps \citet{Brichta07} and \citet{Bocharova11} happened to find an optimal set of geometries to form their inital simplex used to produce figures \ref{fig:simplexJPhysB} and \ref{fig:simplexPRL}, however no mention of it is made, casting some doubt over their geometry reconstructions and any consequent conclusions in their respective works. The neccessity of fine-tuning the Nelder-Mead method should immediately prompt the search for an improved method.

\section{Exploratory data analysis of the momentum measurements}
Before we attempt to reconstruct geometries based on the momentum vectors measured, it would be a good idea to inspect and analyze the raw data first. Such an analysis will help us identify any issues with the data and really we should be making sense of the raw data before attempting to analyze it further. Such an analysis is typically termed exploratory data analysis, first promoted by \citet{Tukey77}.

For each Coulomb explosion event, our measurements for \ch{OCS} include a momentum vector $(p_x, p_y, p_z)$ for each atom or fragment. Thus each event involves the measurement of $9$ scalars, ultimately forming a $9$-dimensional multivariate dataset with some structure, mainly imposed by the condition that momentum must be conserved along each physical axis in the reference frame of the molecule, that is the center-of-momentum (COM) frame for us.

For this section we will look at the momentum vectors measured for \ch{OCS} in the $(2,2,2)$ fragmentation channel following Coulomb explosion by a \SI{7}{\fs} laser pulse, with all measurements kept in the lab frame. This allows us to inspect the raw data before it is transformed to the COM frame with the added benefit that it allows for the inspection of any instrumental or systematic errors. Qualitatively, the data looks quite similar in the two frames, but the geometry reconstruction can only be done in the COM frame as the simulation assumes that the molecule starts from rest such that momentum is always conserved as the molecular systems undergoes a Coulomb explosion.

In figure \ref{fig:OCS2227fsMomentum} we plot histograms for each atom's momentum components and overlay the histograms with kernel density estimates (using a Gaussian kernel) to estimate the probability density function of each momentum component. Glancing at the distributions, they seem to make sense qualitatively with some pecularities as discussed in the caption. 

This analysis should be done for each data set collected. So for us, it should be repeated for the 30, 60, 100, and \SI{200}{\fs} data sets which we include in appendix \ref{appx:supplementaryFigures} as figures \ref{fig:OCS22230fsMomentum}--\ref{fig:OCS222200fsMomentum}.

\begin{figure}
  \centering
  \includegraphics[width=\textwidth]{Plots/OCS2227fsMomentum}
  \caption[OCS (2,2,2) 7fs momentum distributions.]
  {OCS (2,2,2) \SI{7}{\fs} momentum distributions. Distributions for each atom's momentum components measured after Coulomb explosion by a \SI{7}{\fs} laser pulse for the $(2,2,2)$ fragmentation channel (in the lab frame). We can infer some basic dynamics from these plots. For example, we see that the carbon and sulfur atoms tend to fly off with high velocity in the $x$-direction while the oxygen atom flies off with little velocity in the $x$-direction. As conversation of momentum must hold, the $p_x$ components must sum to zero suggesting that the carbon and sulfur tend to fly off in opposite $x$-directions. The other distributions do not tell us much except for the slight asymmetry in the carbon and sulfur's $p_z$ distributions. We also notice that the momentum distribution is rather isotropic in the $y$ and $z$ directions but is bimodal in the $x$ direction, suggesting that the laser pulse's electric field was polarized in the $x$-axis. To inspect this 9-dimensional distribution in more detail we look to figure \ref{fig:OCS2227fsMomentumPairPlots}. Kernel density estimates (with a Gaussian kernel) are used to estimate the probability density function of each momentum component. Such estimates are inherently not as accurate at estimating bimodal distributions thus the carbon and sulfur's $p_x$ estimates appear oversmoothed.}
  \label{fig:OCS2227fsMomentum}
\end{figure}

While figure \ref{fig:OCS2227fsMomentum} gives us some insight into the measurements, we are looking at a multidimensional data set and might want to look at correlations between each measurement. We do this using a pairs plot, which is a grid of scatterplots showing the bivariate relationships between all pairs of variables in a multivariate dataset. The pairs plot was first introduced by \citet{Hartigan75}, however we use the more modern and generalized version introduced by \citet{Emerson13} as they provide an open-source implementation in the form of an R package.

Figure \ref{fig:OCS2227fsMomentumPairPlots} uses a pairs plot to showcase the relationship between ever pair of variables in our dataset. Due to the $9$-dimensional nature of our dataset, we end up with a $9\times9$ grid of plots with some redundancy. The diagonal repeats the kernel density estimates shown in figure \ref{fig:OCS2227fsMomentum} but they are quite useful as a reference here. Below the diagonal are the scatter plots, however due to the high density of points, contour plots are employed to showcase the same relationship above the diagonal. So only 36 scatter plots are required but this format provides us with greater insight of our dataset.

\begin{figure}
  \centering
  \includegraphics[width=\textwidth]{Plots/OCS2227fsMomentumPairsPlot}
  \caption[Pairs plot showing the bivariate relationship between each atom's momentum components measured after Coulomb explosion by a \SI{7}{\fs} laser pulse for the $(2,2,2)$ fragmentation channel.]
  {Pairs plot showing the bivariate relationship between each atom's momentum components measured after Coulomb explosion by a \SI{7}{\fs} laser pulse for the $(2,2,2)$ fragmentation channel. On the diagonal, kernel density estimates (with a Gaussian kernel) of the momentum component designated by the label at the top of the column and end of the row are given. Below the diagonal, scatter plots show the relationship between the momentum components belonging to that row and column. Above the diagonal, the same relationship is given using a contour plot instead. For example, the scatter plot in row $7$, column $1$ plots the oxygens's $p_x$ component on the $x$-axis and the sulfur's $p_x$ component on the $y$-axis. The contour plot in row $1$, column $7$ shows the same relationship. We see a negative correlation between oxygen's $p_x$ and sulfur's $p_x$ as predicted in figure \ref{fig:OCS2227fsMomentum}'s caption. Similarly we see negative correlations between oxygen's $p_y$ and sulfur's $p_y$ as well as between oxygen's $p_z$ and sulfur's $p_z$, which makes physical sense due to the two atoms being terminal atoms, so they should fly off in approximately opposite directions following a Coulomb explosion.}
  \label{fig:OCS2227fsMomentumPairPlots}
\end{figure}

Pairs plots for the 30, 60, 100, and \SI{200}{\fs} data sets are included in appendix \ref{appx:supplementaryFigures} as figures \ref{fig:OCS22230fsMomentumPairPlots}--\ref{fig:OCS222200fsMomentumPairPlots}.

\section{Implementation}
\index{Lookup table}
\index{Geometry reconstruction!Lookup table}
The idea behind using a lookup table is quite simple: many Coulomb explosions are simulated (see secton \ref{sec:simulating}) using a wide variety of molecular structures as the initial condition, and the resulting momentum vectors from each simulation are stored. This creates a mapping from molecular structures to momentum vectors. Each molecular geometry and its corresponding post-explosion momentum vectors can be stored in a single row consisting of $9$ entries, $3$ to describe the molecular geometry of a triatomic molecule and $6$ to describe the momentum vectors (see section \ref{sec:conventions}). While only $3$ scalars are enough to describe the momentum vectors produced by a simulated Coulomb explosion (the last momentum vector can be calculated from the others using conservation of momentum), we store all $6$ to allow for comparison with real momentum data where throwing out some components seems somewhat sketchy. Thus by storing the results from many simulations, we end up with a lookup table. To determine the structure belonging to a certain set of observed momentum vectors, you simply read the table in reverse and find the momentum vectors that most closely match the observed set, that is minimize the $2$-norm between the sets of vectors.

This approach is simple to implement, very quick by design, and front-loads the computation which may be desirable for large data sets. However, of course, it has an exponential time and space complexity $\mathcal{O}(e^{3N-6})$ where $N$ is the number of atoms.

One way to increase the potential accuracy of this method is to 

For these reasons, the lookup table was abandoned rather quickly in favor of a mathematical optimization approach.

\subsection{Using simulations to test accuracy}
It is important that we do not simulate geometries that are already contained in the lookup table.

\section{Geometry reconstructions done by lookup table}

\pagebreak
\begin{figure}
  \centering
  \includegraphics[width=\textwidth]{Plots/OCS2227fsLTGeometry}
  \caption[OCS (2,2,2) \SI{7}{\fs} geometry reconstruction by lookup table.]
  {OCS (2,2,2) \SI{7}{\fs} lookup table geometry. A scatter plot showing the molecular geometry of \ch{OCS} following Coulomb explosion by a \SI{7}{\fs} laser pulse for the $(2,2,2)$ fragmentation channel. Each individual geometry is represented by three points, one for each atomic fragment. Geometries are plotted such that the molecule's center of mass is at the origin to showcase the variance in each atomic fragment's position. Bivariate kernel density estimates (with a Gaussian kernel), plotted as shaded-in contours, are used to estimate the the probability density of each atomic fragment's position. Solid black lines are drawn between the peaks of each atomic fragment's kernel density estimate to illustrate the modal geometry or most likely geometry. Along the top axis, marginal distributions show the probability density of each  atomic fragment's position along the $x$-axis and the same is done for the $y$-axis along the right. The modal geometry is calculated to be $r_\mathrm{CO} = \SI{1.32}{\angstrom}$, $r_\mathrm{CS} = \SI{1.26}{\angstrom}$, $ \theta_\mathrm{OCS} = 172.7\degree$ while the average geometry of $r_\mathrm{CO} = \SI{1.39}{\angstrom}$, $r_\mathrm{CS} = \SI{1.27}{\angstrom}$, $ \theta_\mathrm{OCS} = 171.6\degree$ is slightly larger due to outliers. In chapter \ref{ch:optimization} we will see that plotting geometries in this manner, while intuitive and satisfying, can be very misleading.}
  \label{fig:OCS2227fsLTGeometry}
\end{figure}

The calculated modal and average geometries are quite worrying.

A critical issue is quantifying how much uncertainty there is in determining these molecular geometries. This is a complicated problem we attempt to tackle in chapter \ref{ch:bayesian}.

It probably does not matter how geometries are plotted. In chapter \ref{ch:optimization} we will see that plotting geometries in this manner, while intuitive and satisfying, can be very misleading. This is because the ensemble of geometries reconstructed seems to make physical sense while most of the individual geometries reconstructed do not.

\section{Conclusions and lessons learnt}
\subsection{Precision is computationally expensive}
\subsection{Difficulty in scaling to larger molecules}

\subsection{Degenerate molecular structures}
We have previously hinted at the fact that due to the ill-posed nature of the geometry reconstruction inverse problem, multiple solutions may be posible. This feature actually does pop up for the OCS molecule and the lookup table is perhaps the best method of investigating these multiple solutions, which we will call degenerate geometries.

\begin{figure}
  \centering
  \includegraphics[width=\textwidth]{Plots/DegenerateGeometryTrajectories.pdf}
  \caption[Atomic trajectories in position and momentum-space, and kinetic energy, of two degenerate molecular geometries undergoing a Coulomb explosion.]
  {Atomic trajectories in position and momentum-space, and kinetic energy, for the oxygen, carbon, and sulfur atoms of two degenerate \ch{OCS} molecular geometries in the $(2,2,2)$ charge state undergoing a Coulomb explosion staring from rest. The plots on the left correspond to an \ch{OCS} molecule with $r_\textrm{CO} = \SI{1.8949}{\angstrom}$, $r_\textrm{CS} = \SI{1.2990}{\angstrom}$, and $\theta_\mathrm{OCO} = \SI{160.601}{\degree}$ while the plots on the right correspond to an \ch{OCS} molecule with $r_\textrm{CO} = \SI{2.486}{\angstrom}$, $r_\textrm{CS} = \SI{1.0755}{\angstrom}$, and $\theta_\mathrm{OCO} = \SI{164.568}{\degree}$. The molecular geometries are quite different yet when they undergo a Coulomb explosion, they produce the exact same set of momentum vectors once rotated into our convention (see section \ref{sec:conventions}). Well, we can make the difference (or $2$-norm) between the two sets of momentum vectors arbitrarily small with increased precision. Thus we call these two molecular geometries \emph{degenerate}. We see that the molecular dynamics are a little different yet in both cases, all three atoms emerge with the exact same kinetic energy.}
  \label{fig:DegenerateGeometryTrajectories}
\end{figure}
\clearpage

\section{Future usefulness}

% Another warning box that zooming in doesn't actually help much. If you're
% near a local minimum, you'll just zoom into that. A first go usually gives you ~ 2 sig figs of precision, anything more doesn't mean much as our momentum uncertainty is pretty high.
% Warning awesomebox that the MATLAB code has not been vectorized.