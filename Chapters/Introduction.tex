\chapter{Introduction} \label{ch:introduction}

%\vspace{-1.5 em}
%\begin{addmargin}[-0.5cm]{0cm}
%  \minitoc
%\end{addmargin}
%\hrule
%\vspace{1.5 em}

To image the structure and dynamics of a molecule by destroying it seems paradoxical, yet that is precisely what Coulomb explosion imaging (CEI) does. Exposing a molecule to the intense electric field of an ultrashort laser pulse causes it to rapidly ionize and dissociate into its constituent atomic fragments, in the case of complete dissociation, termed a \emph{Coulomb explosion}. The momentum vectors of the ``atomic shrapnel'' contain a wealth of information about the molecule and its dissociation, thus by measuring them the molecule's original geometry may be reconstructed. This is the basic idea behind CEI. The laser pulse may be used to induce specific molecular dynamics then dissociation at particular times, either through the temporal lengthening of the pulse or the use of a pump then a probe pulse, which CEI can image forming a series of frames constituting a \emph{molecular movie}.

\section*{Motivation and research problem}
A cursory review of CEI research articles would suggest that the reconstruction of molecular geometries is a well-understood process. For example, \citet{Xu16} writes, ``CEI is an effective and straightforward tool to retrieve molecular structural properties, requiring no prior assumptions about the molecule'', and \citet{Matsuda14} states that ``Coulomb explosion provides a direct access to the instantaneous structure of the target molecule'' \citep{Matsuda14}. \citet{Vager01}, in an older review article of CEI, writes that ``simultaneous determination of the final fragments velocity vectors from each single molecule, which is what CEI detectors do, resolves the 3D initial conformation'', however the article contains no molecular geometries or structures. This is true of the majority of research articles employing CEI which tend to study the momentum vectors themselves, inferring molecular dynamics and structure from the kinetic energy spectra of the atomic fragments, and from the arrangement of the vectors through the use of Newton plots and Dalitz plots \citep{Ramadhan16}. For the few that do attempt to determine the molecular structures, the methodology is vague, sometimes even reduced to a single sentence, such as ``we find a three-dimensional structure that reproduces the measured fragment velocities'' \citep{Legare05structure}.

In this thesis, we will attempt to address this knowledge gap by developing a computational framework for geometry reconstruction using CEI, with the aim of providing fast, accurate reconstructions and quantifying the uncertainty on these geometry reconstructions. The problem of geometry reconstruction may be thought of as an ill-posed \emph{inverse problem} with no analytic solution or iterative process that guarantees a solution. However, inverse problems in science are routinely tackled effectively using mathematical optimization techniques, which we employ, and Bayesian inference, which we will propose and discuss.

\section*{Thesis outline}
Chapter \ref{ch:CEI} provides a brief overview of CEI including the method's aspirations and an experimental outline as well as a review of past geometry reconstruction attempts.

Chapter \ref{ch:data} details the multi-step process of measuring the momentum vectors for each atomic fragment following a Coulomb explosion, and the quantification of measurement uncertainty for the momentum vectors. We will also perform some exploratory data analysis of the measurements then discuss how to computationally simulate a Coulomb explosion using Hamiltonian mechanics. Finally, we will detail some necessary conventions for describing molecular geometries and momentum vectors.

Chapters \ref{ch:lookupTable} and \ref{ch:optimization} constitute the main portion of this thesis, detailing the two different approaches taken to geometry reconstruction. Chapter \ref{ch:lookupTable} discusses the lookup table approach and its motivation to supersede a previous approach relying on the Nelder-Mead simplex method. A lookup table is implemented and used to perform geometry reconstruction using the \ch{OCS} molecule as an example, and to study the existence of \emph{degenerate geometries}. Chapter \ref{ch:optimization}, motivated by the drawbacks of the lookup table, formulates the task of geometry reconstruction as an optimization problem, which is tackled using nonlinear constrained optimization algorithms. Some theory from mathematical optimization is introduced to understand the methods employed, and an implementation in MATLAB is used to reconstruct the \ch{OCS} molecule and the two reconstructions are compared. The optimization approach also allows for the further investigation of degenerate geometries.

Chapter \ref{ch:uncertainty} addresses the important task of uncertainty quantification for reconstructed geometries, which has not been addressed by previous studies. A heuristic approach is employed at first and used to provide insights into the effects of uncertainty. A more rigorous and sophisticated approach in the Bayesian inference framework is discussed and suggested as a next step.
