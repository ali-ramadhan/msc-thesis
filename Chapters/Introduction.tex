\chapter{Introduction}\label{ch:introduction}

To image the dynamics of a molecule by destroying it seems paradoxical. As we shall see, the molecule's structure is encoded in the atmoic shrapnel left behind after an explosion. However, to destroy one of nature's simplest creations is no easy task. Molecules are held together by strong chemical bonds. Our best line of attach is to shoot them with a short laser pulse---engulfing the molecule in an intense oscillating electric field will strip away its electrons and cause it to break up quickly. In these first few chapters, I will discuss how to create a short laser pulse and how these pulses interact with matter. Then I will introduce the technique of pump-probe Coulomb explosion imaging, which we will use to probe the dynamic structure of small molecules by studying the atomic fragments resulting from the explosion.

\section{Motivation}
The scope of this imaging method is to make measurements of the geometries of single molecules on a timescale faster than that of molecular motion ($10^{-15}$ seconds). Ultimately these images can be recorded in sequence to image the dynamics of a molecule. It works well for small molecules in the gas phase, a regime in which no other method has been viable.

The method involves shooting a molecule in a vacuum chamber with two laser pulses, a pump pulse followed by a probe pulse after some time delay $\tau \ge 0$. The pump causes the molecule to undergo some change, then the more intense probe pulse strips off enough electrons from the molecule such that the molecule's individual atoms separate and repel each other ``explosively'' due to Coulomb repulsion. The further apart the atoms are repelled, the weaker the repulsion and eventually every atom reaches an asymptotic state of constant velocity. This all takes place in a constant electric field that accelerates all the atoms towards a time and position-sensitive detector. The detector can tell us how much time each atom took to reach it, and the position where the atom hit it. From this information, the atom’s (asymptotic) momentum can be calculated and then provided we know how to simulate the experiment in reverse, we can determine the initial geometry of the molecule just before it was hit by the pump pulse. By picking different values of $\tau$ and repeating the experiment many times for each $\tau$ we get a ``frame'' of the molecule's geometry at each $\tau$ , which may be placed in sequence to form a ``molecular movie'' of the change induced by the pump pulse.

\section{Research problem}
Determining these geometries is a computationally difficult task. Almost every study dodges the problem entirely and relies on the momentum vectors to infer information about the geometries. The momentum vectors carry a lot of information but that is certainly not the structural imaging promised by Coulomb explosion imaging that is talked about. It also does not capture all the dynamics possible. When geometry reconstruction is done, it is usually done in a hand wavy way that sidesteps the nuances of the problem and ignores uncertainty, rendering it unreliable. I am working on a rigorous method to perform the geometry reconstruction that may be used (and improved) by my group and others.

\section{Approach taken}
We are only interested in the molecular geometry, i.e. the relative positions of the atoms, not the absolute position of every atom. This reduces the dimensionality of our problem from $3n$ to $3n-6$ in both the geometry $\mathbf{g}$ and momentum vector $\mathbf{p}$. The geometry ``vector'' $\mathbf{g} \in \mathbb{R}^{3n-6}$ contains the molecule's bond lengths and bond angles, and the momentum ``vector'' $\mathbf{p} \in \mathbb{R}^{3n-6}$is a concatenation of momentum values in a specific convention. They both do not transform like vectors.

Having destroyed a molecule and measured the momentum of each of its atomic fragments, we are left with the inverse problem of inferring its structure. While explosions proceed in a deterministic fashion, that is structures map bijectively to momentum measurements, the converse is not true. Two very different structures may produce the same momentum measurements. To make matters worse, there is no analytic solution to the problem and as the molecule grows, the problem of finding its structure becomes increasingly high dimensional. To combat this problem we will require the use of various mathematical and statistical methods. I first discuss some results from the theory of inverse problems to shed some general insight on these problems. I then follow with a discussion of optimization methods which may be used to tackle the problem for very small molecules. However, for full imaging of larger polyatomic molecules with an analysis of measurement error, Bayesian inference using Markov chain Monte Carlo methods is the way to go, which I discuss in the last chapter of this part.

To find the initial geometry $\mathbf{g}_0$ that produced the measured asymptotic momentum $\mathbf{p}_\infty$, we make use of the fact that simulating the Coulomb explosion forward in time is easy. Let $\mathbf{p}(\mathbf{g}) : \mathbb{R}^{3n-6} \rightarrow \mathbb{R}^{3n-6}$ map an initial geometry $\mathbf{g}$ onto the asymptotic momentum vector $\mathbf{p}_\infty$ such a geometry produces upon Coulomb explosion. It simulates the explosion forwards in time by solving a set of $6n-12$ coupled first-order ODEs. Then casting this as an optimization problem, we seek the geometry $\mathbf{g}$ that minimizes the objective function $\left\| \mathbf{p}(\mathbf{g}) - \mathbf{p}_\infty \right\|_2^2$.

\section{Thesis outline}

Chapter \ref{ch:history} will provide a brief history of molecular movies and geometry reconstruction attempts, with a particular focus on attempts employing Coulomb Explosion imaging (CEI). The history and achievements of CEI will also be briefly reviewed.

Chapter \ref{ch:CEI} will provide a more detailed description of CEI including the experimental schemes generally employed, such as pump-probe CEI and Femtosecond Multiple Pulse Length Spectroscopy (FEMPULS). Focusing on the experimental apparatus used for this work, we will also go through the process of how data is collected for each atmoic fragment following a Coulomb explosion event. We will then discuss how to computationally simulate a Coulomb explosion using Hamiltonian mechanics, which is an essential step in reconstructing molecular geometris. Finally we will discuss some neccessary conventions for describing molecular geometries and momentum vectors.

Chapters \ref{ch:lookupTable}-\ref{ch:bayesian} constitute the main portion of this thesis as they detail the three different approaches taken to geometry reconstruction. Chapter \ref{ch:lookupTable} discusses the somewhat limited lookup table approach, while chapter \ref{ch:optimization} formulates the task as an optimization problem with some success in making point estimates of the geometries, and chapter \ref{ch:bayesian} uses Bayesian inference to infer the distribution of the molecular geometries taking measurement uncertainty into account.