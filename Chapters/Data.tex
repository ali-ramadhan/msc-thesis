\chapter{Data measurement and analysis} \label{ch:data}

\vspace{-1.5 em}
\begin{addmargin}[-0.5cm]{0cm}
  \minitoc
\end{addmargin}
\hrule
\vspace{1.5 em}

The premise of CEI is simple enough for a quirky elevator pitch yet the collection and analysis of data is a nuanced multi-step process as we are attempting to measure the momentum vector of each single atomic fragment precisely. This chapter will discuss the data measurement and analysis process in detail from the measurement of position and time to the calculation of the momentum vector components and theoretical lower-bound estimates of the measurement uncertainty. We will explore the collected data set as well as provide some information regarding the simulation of a Coulomb explosion and the conventions used throughout this thesis for describing the geometries and momentum vectors.

\section{Data measurement} \label{sec:measurement}
In this section we will go through the process of how the momentum vectors of each atomic fragment are measured. This will require some discussion regarding the apparatus, algorithms, and intricacies of the process, all of which are essential to understand exactly how the data is collected so that it can be analyzed appropriately. We will also quantify the uncertainty in those measurements, which will be useful in quantifying our uncertainty in the reconstructed geometries in chapter \ref{ch:uncertainty}.

\subsection{Time and position measurement}
The time-of-flight of each atomic fragment and its position on the detector are required to calculate its momentum vector. The measurement of time and position is carried out by a two-stage apparatus feeding electrical signals into a data acquisition (DAQ) computer which analyzes the signals to determine time and position.

The first part of the two-stage apparatus is a set of two micro-channel plates (MCP) placed in a chevron configuration.\footnotemark~ Figure \ref{fig:MCP} shows a schematic of an MCP. Once a charged particle is incident on an MCP and collides with a channel wall, multiple secondary electrons are emitted and accelerated up the channel due to the applied voltage $V_D$ setting up a potential gradient along the channel and replenishing the emitted electrons. Due to the angled channels, the emitted electrons follow parabolic trajectories hitting the other wall and continuing the amplification process until a large number of particles ($\sim10^4$, depending on the applied voltage) are emitted at the channel output.

\footnotetext{The channels of an MCP are slanted, usually at a (bias) angle of \SIrange{5}{15}{\degree} to increase the probability that an incident particle collides with the channel wall. To further increase this probability, the second MCP is oriented such that its channels are slanted in the opposite direction forming a V-like (or chevron-like) channel configuration.}

\begin{SCfigure}
  \centering
  \includegraphics[width=0.50\textwidth]{gfx/MCP}
  \caption
  [Schematic of a multi-channel plate (MCP).]
  {Schematic of a multi-channel plate (MCP). The incident particle need not be an electron, and may in fact be any charged particle, a high-energy photon, or a neutrally charged particle with kinetic energy larger than the work function of the glass.}
  \label{fig:MCP}
\end{SCfigure}

The job of the MCP is to amplify the signal of a single charged particle enough such that it may be detected as an electrical signal by an oscilloscope, much like a photomultiplier tube. Thus the output of an MCP is a shower of charged particles, or rather a charged cloud. The charged cloud may be fed into a second MCP to further amplify the signal. A two-stage chevron MCP setup produces an amplification of approximately $10^5$ to $10^7$ depending on the applied voltage $V_D$ across the channels.

You may notice that while most of the MCP's surface is covered in channels, not all of it is, leading to an imperfect detection rate. Only 60\% of the area is open to incident particles, and if a particle is incident on the other 40\% then it is not detected. Thus the detection efficiency of a triple coincidence event is $(0.6)^3 \approx 0.2$ and so we see that detection efficiency decreases rapidly with the number of fragments that must be detected, suggesting that larger molecules are more difficult to study. There do exist ``funnel'' MCPs with an open area ratio of 90\% that increase the detection efficiency.

By itself this MCP setup is enough to provide time-of-flight information but to obtain position information, this charged cloud output is made incident on a position sensitive anode such as a ``modified backgammon with weighted capacitors'' anode or readout pad built as described by \citet{Veshapidze02} and used by \citet{Ramadhan16}. Figure \ref{fig:MBWC} shows a schematic of such an anode.

\begin{figure}
  \centering
  \includegraphics[width=\textwidth]{gfx/MBWC}
  \caption
  [Schematic of a symmetrized ``modified backgammon with weighted capacitors'' (MBWC) anode for position detection.]
  {Schematic of a symmetrized ``modified backgammon with weighted capacitors'' (MBWC) anode for position detection. Figure from \citet{Mizogawa02}. Reprinted with permission from Elsevier.}
  \label{fig:MBWC}
\end{figure}

The avalanche of charged particles, or the charged cloud, hits the anode and induces a charge on it. This charged is induced via the capacitive couplings from the feedback capacitors of the preamplifiers connected to the triangular conducting strips. The lines on the anode are insulating gaps, splitting the anode into a series of triangles whose arrangement resemble that of the backgammon board game. The metal strips are capacitively coupled to the triangular strips through the insulator. If the cloud lands on the right side of the anode, then a larger fraction of the induced charge will flow to $Q_3$ and $Q_4$. So we can see that $x=0$ corresponds to the left side of the anode, and $x=1$ to the right side. If the cloud lands further up the anode, then a larger fraction of the induced charge will flow through $Q_1$ and $Q_3$ so we see that $y=0$ corresponds to the bottom side of the anode and $y=1$ corresponds to the top side. The design gets its name as it is a combination of two older designs, the ``backgammon'' (BG) and the ``weighted coupling capacitor'' (WCC) designs. \citet{Mizogawa92} provides a more detailed explanation of its operation. This setup provides position information on a scale $0 \le x,y \le 1$ and must be multiplied by appropriate scaling factors for each dimension, which depend on the physical dimension of the setup, to retrieve the correct position information in meters.

These four $Q_i$ signals are subsequently fed into an Ortec 142 preamplifier in energy output mode, essentially acting as an operational amplifier integrator. The integrated signal is then fed into a data acquisition (DAQ) computer equipped with a four-channel oscilloscope. Every time a laser pulse is fired into the experimental apparatus, an electrical signal is sent to the oscilloscope as a trigger. The software running on the DAQ computer examines the four signals following a trigger and saves them if it detects evidence of charged particle detection, using a coincidence detection algorithm for improving detection rates as described by \citet{Wales12Algorithm}.

A particularly good example of a triple coincidence detection event can be seen in figure \ref{fig:tripleCoincidence}, wherein the fragmentation event showcased is the concerted breakup process \ch{CS2 $\rightarrow$ CS2^3+ $\rightarrow$ C^+ + S^+ + S^+}.

\begin{figure}
  \centering
  \includegraphics[width=\textwidth]{gfx/TripleCoincidenceEvent}
  \caption
  [Spectrometer response during a triple coincidence event.]
  {Spectrometer response during a triple coincidence event. This was taken during a CEI experiment studying the dynamics of \ch{CS2} at the Canadian Light Source. The fragmentation event showcased is the concerted breakup process \ch{CS2 $\rightarrow$ CS2^3+ $\rightarrow$ C^+ + S^+ + S^+}.}
  \label{fig:tripleCoincidence}
\end{figure}

As the carbon atom is lighter, the first step at \SI{500}{\ns} is the detection of the carbon atom. All four channels increase by roughly similar amounts hinting that the carbon atom was detected near the center of the detector by \eqref{eq:xy}. Then the second and third events belong to the two sulfur atoms, arriving later due to sulfur's larger atomic mass. If the molecule was aligned in a plane parallel to the detector, and both sulfur atoms started with zero initial momentum, then they would have remained at the same height throughout the Coulomb explosion and been detected at the same time, producing one step in the signal. However, one must have had some downward initial momentum with the other having some upward initial momentum, which may be a result of the molecule having been oriented vertically such that one sulfur atom was closer to the detector. During the Coulomb explosion, the closer atom will initially experience a kick towards the detector while the other sulfur atom will initially experience a kick away from the detector before being accelerated upwards due to the constant electric field. This results in one sulfur atom arriving earlier, and the other later. Looking at the individual signals, we see a significant increase in $Q_1$ and $Q_2$ at \SI{800}{\ns} suggesting that the first sulfur atom was on one side of the detector while the more significant increase in $Q_3$ and $Q_4$ at \SI{900}{\ns} suggest that the second sulfur atom was on the other side. This makes some intuitive sense as we expect the carbon atom to land somewhere near the middle and the two sulfurs to land on opposite sides of the detector. The oscilloscope cards sport an 8-bit bus and so the individual $Q_i$ channels were limited to \SI{200}{\mV} to increase position detection accuracy. Triple coincidence events tend to make up approximately $(0.6)^3 \approx 0.2$ of detected events due to the detector's imperfect detection rate, however rich and interesting signals such as this one can be rarer. The majority of events detected tend to be single or double coincidences of course.

As a side note, if too many atomic fragments arrive at the detector in a short enough time period, steps due to different molecules may get mixed. The laser typically produces pulses with a repetition rate of \SI{1}{\kilo\Hz} which will keep different events from overlapping. When the experiment is carried out at a synchrotron facility, adjusting the light intensity until a lower event rate is observed (\SI{100}{\Hz}) works better due to the lower level of control we have over the event rate \citep{Ramadhan16}. Another reason to keep the count rate low is that the signals need time to decay back down post-integration otherwise the signals will saturate the oscilloscope at \SI{200}{\mV}. The ringing artifacts on the signal are due to impedance mismatches between the anode and preamplifiers.

Software can analyze the signals and determine the magnitude of each $Q_i$ signal. If the change in baseline before and after an event is denoted $Q_i'$ then the position of the electron is then calculated using
\begin{equation}\label{eq:xy}
x = \frac{Q_1' + Q_2'}
{Q_1' + Q_2' + Q_3' + Q_4'} ,\quad
y = \frac{Q_1' + Q_3'}
{Q_1' + Q_2' + Q_3' + Q_4'}
\end{equation}
where $x,y \in [0,1]$ are fractional positions. Multiplying $x$ and $y$ by the dimensions of the MCP detector will yield the physical position of the cloud's centroid.

\subsection{Calculating the atomic fragments' momenta}
Calculating the momentum vector of each atomic fragment is an elementary physics problem once we have the time and position measurements. The components of the three-dimensional momentum vector $\mathbf{p} = (p_x,p_y,p_z)$ for each atom are calculated as
\begin{equation}\label{eq:CEImomenta}
p_x = \frac{m(x-x_0)}{t} ,\;
p_y = \frac{m(y-y_0)}{t} ,\;
p_z = \frac{qV}{2\ell} \left( \frac{t_0^2 - t^2}{t} \right)
\end{equation}
where $m$ is the atom's mass, $(x,y)$ is the location the atom collided with the MCP detector, and $(x_0,y_0)$ is the location that the Coulomb explosion originated. The location $(0.5,0.5)$ corresponds to the physical center of the MCP detector. $q$ is the net charge of the atom, $V$ is the value of the constant electric field the atom is subjected to, and $\ell$ is the distance from the location of the Coulomb explosion to the detector. $t$ is measured time of flight (between Coulomb explosion and detection) of the atom and 
\begin{equation}
t_0 = \sqrt{\frac{2d\ell}{V} \left( \frac{m}{q} \right)}
\end{equation}
is the atom's time of flight assuming no external forces act on it during its trip to the detector.

\subsection{Measurement uncertainty in the momenta} \label{ssec:measurementUncertainty}
For any relation $f = f(x_1, x_2, \dots, x_n)$, assuming independent variables (neglecting correlations), the standard deviation (or absolute uncertainty) in a quantity $f$, which we denote $\Delta f$, may be calculated using the variance formula \citep{Ku66}, which has been very popular among physical scientists,
\begin{equation} \label{eq:varianceFormula}
\Delta f = \sqrt{\sum_{i=1}^{n} \left( \frac{\partial f}{\partial x_i} \Delta x_i \right)^2}
\end{equation}
where $\Delta x_i$ is the standard deviation in the independent variable $x_i$ and where the partial derivatives $\partial f/\partial x_i$ are evaluated at the mean of $x_i$. This formula relies on the linear characteristic of the gradient of $f$ and so it's a good estimate for the standard deviation of $f$ as long as the standard deviations $\Delta x_i$ are small compared to the partial derivatives. However, as it employs a truncated Taylor series, it may even be a biased estimate in some cases. Much can be said about which formula to use in the propagation of uncertainty, and expository articles on the subject have been written by \citet{Birge39} and \citet{Ku66}, the latter of which provides an insightful derivation of \eqref{eq:varianceFormula}. In our case, we will not attempt to make accurate nor precise calculations of any uncertainty, so we will not fuss about which method we choose to propagate our uncertainties forward. We are simply interested in making order-of-magnitude estimates of the uncertainties on the measured momentum vectors in order to make order-of-magnitude estimates on the uncertainty of reconstructed geometries in chapter \ref{ch:uncertainty}.

Using \eqref{eq:varianceFormula} we may calculate the uncertainty in the measured momentum values, which will be different for each component. In our case, $p_x$ is a function $p_x(m,x,x_0,t)$ and $p_y = p_y(m,y,y_0,t)$, however, the uncertainty in the atomic mass $m$ is orders of magnitude smaller than the uncertainty in the other variables and so we will ignore its effects. Thus we get that
\begin{subequations}
  \begin{align}
  \Delta p_x &= \sqrt{
    \left( \frac{\partial p_x}{\partial x}\Delta x \right)^2
    + \left( \frac{\partial p_x}{\partial x_0}\Delta x_0 \right)^2
    + \left(\frac{\partial p_x}{\partial t}\Delta t \right)^2
  } \\
  \Delta p_y &= \sqrt{
    \left( \frac{\partial p_y}{\partial y}\Delta y \right)^2
    + \left(\frac{\partial p_y}{\partial y_0}\Delta y_0 \right)^2
    + \left(\frac{\partial p_y}{\partial t}\Delta t \right)^2
  }
  \end{align}
\end{subequations}
where the partial derivatives can be calculated from \eqref{eq:CEImomenta} as
\begin{subequations}
  \begin{align}
  \frac{\partial p_x}{\partial x} = \frac{m}{t} &,\quad \frac{\partial p_x}{\partial x_0} = \frac{m}{t} ,\quad \frac{\partial p_x}{\partial t} = -m\frac{x-x_0}{t^2}\\
  \frac{\partial p_y}{\partial y} = \frac{m}{t} &,\quad \frac{\partial p_y}{\partial y_0} = \frac{m}{t} ,\quad \frac{\partial p_y}{\partial t} = -m\frac{y-y_0}{t^2}
  \end{align}
\end{subequations}
and so after plugging in the partial derivatives and rearranging slightly we get that
\begin{subequations}
  \begin{align}
  \frac{\Delta p_x}{p_x} &= \sqrt{
    \left( \frac{\Delta x}{x - x_0} \right)^2
    + \left( \frac{\Delta x_0}{x - x_0} \right)^2
    + \left( \frac{\Delta t}{t} \right)^2 } \\
  \frac{\Delta p_y}{p_y} &= p_y \sqrt{
    \left( \frac{\Delta y}{y - y_0} \right)^2
    + \left( \frac{\Delta y_0}{y - y_0} \right)^2
    + \left( \frac{\Delta t}{t} \right)^2 }
  \end{align}
\end{subequations}

Repeating the process for $p_z = p_z(q,V,\ell,t_0,t)$ but ignoring the tiny uncertainties in $q$, $V$, and $\ell$, we get
\begin{equation}
\frac{\Delta p_z}{p_z} = p_z \sqrt{
  \left( \frac{2tt_0}{t_0^2 - t^2} \Delta t_0 \right)^2
  + \left( \frac{t^2 + t_0^2}{t(t_0^2 - t^2)} \Delta t^2 \right)^2
}
\end{equation}

To calculate the uncertainties in position, $\Delta x$ and $\Delta y$, we repeat the same process for equations \eqref{eq:xy} to arrive at
\begin{subequations}
  \begin{align}
  \Delta x = \frac{1}{Q_T} \sqrt{ (1-x)^2 (\Delta Q_1^2 + \Delta Q_2^2) + x^2 (\Delta Q_3^2 + \Delta Q_4^2)} \\  
  \Delta y = \frac{1}{Q_T} \sqrt{ (1-y)^2 (\Delta Q_1^2 + \Delta Q_3^2) + y^2 (\Delta Q_2^2 + \Delta Q_4^2)}
  \end{align}
\end{subequations}
where $Q_T = Q_1 + Q_2 + Q_3 + Q_4$. The values of $Q_i$ are digitized with the same resolution thus denoting $\Delta Q_i = \Delta Q$ for $i=1,2,3,4$ yields simpler relations
\begin{subequations}
  \begin{align}
  \Delta x = \sqrt{2} \frac{\Delta Q}{Q_T} \sqrt{(1-x)^2 + x^2} \\
  \Delta y = \sqrt{2} \frac{\Delta Q}{Q_T} \sqrt{(1-y)^2 + y^2}
  \end{align}
\end{subequations}

% Errors due to the size of the laser interaction volume seem to be negligible.

\section{Exploratory data analysis}
Before we attempt to reconstruct geometries based on the momentum vectors measured, it would be a good idea to inspect and analyze the raw data first. Such an analysis will help us identify any issues with the data and really we should be making sense of the raw data before attempting to analyze it further. Such an analysis is typically termed exploratory data analysis, first promoted by \citet{Tukey77}.

\subsection{Momentum measurements}
For each Coulomb explosion event, our measurements for \ch{OCS} include a momentum vector $(p_x, p_y, p_z)$ for each atom or fragment. Thus each event involves the measurement of $9$ scalars, ultimately forming a $9$-dimensional multivariate dataset with some structure, mainly imposed by the condition that momentum must be conserved along each physical axis in the reference frame of the molecule, that is the center-of-momentum (COM) frame.

For this section we will look at the momentum vectors measured for \ch{OCS} in the $(2,2,2)$ fragmentation channel following Coulomb explosion by a \SI{7}{\fs} laser pulse, with all measurements kept in the lab frame. This allows us to inspect the raw data before it is transformed to the COM frame with the added benefit that it allows for the inspection of any instrumental or systematic errors. Qualitatively, the data looks quite similar in the two frames, but the geometry reconstruction can only be done in the COM frame as the simulation assumes that the molecule starts from rest such that momentum is always conserved as the molecular systems undergoes a Coulomb explosion.

In figure \ref{fig:OCS2227fsMomentum} we plot histograms for each atom's momentum components and overlay the histograms with kernel density estimates (using a Gaussian kernel) to estimate the probability density function of each momentum component. Glancing at the distributions, they seem to make sense qualitatively with some peculiarities. 

\begin{figure}
  \centering
  \includegraphics[width=\textwidth]{Plots/OCS2227fsMomentum}
  \caption[Distributions for each atom's momentum components measured after Coulomb explosion by a \SI{7}{\fs} laser pulse for the $(2,2,2)$ fragmentation channel (in the lab frame).]
  {Distributions for each atom's momentum components measured after Coulomb explosion by a \SI{7}{\fs} laser pulse for the $(2,2,2)$ fragmentation channel (in the lab frame). Kernel density estimates (with a Gaussian kernel, see section \ref{sec:kde}) are overlaid to estimate probability distributions.}
  \label{fig:OCS2227fsMomentum}
\end{figure}

We can infer some basic dynamics from these plots. For example, we see that the oxygen and sulfur atoms tend to fly off with high velocity in the $x$-direction while the carbon atom flies off with little velocity in the $x$-direction. As conservation of momentum must hold, the $p_x$ component of the molecular system must sum to zero suggesting that the oxygen and sulfur tend to fly off in opposite $x$-directions. This is expected as they are both terminal atoms and confirms that the data makes physical sense. The other distributions do not say much except for the slight asymmetry in the oxygen and sulfur's $p_z$ distributions which may be due to instrumental bias in measuring the arrival time of atomic fragments. We also notice that the momentum distribution is rather isotropic in the $y$ and $z$ directions but is bimodal in the $x$ direction, suggesting that the laser pulse's electric field was polarized along the $x$-axis. Kernel density estimates (with a Gaussian kernel) are used to estimate the probability density function of each momentum component. Such estimates are inherently not as effective at estimating bimodal distributions thus the carbon and sulfur's $p_x$ estimates appear over smoothed.

This analysis should be done for each data set collected. In our case, this includes the \SIlist{30;60;100;200}{\fs} data sets which we include in appendix \ref{appx:supplementaryFigures} as figures \ref{fig:OCS22230fsMomentum}--\ref{fig:OCS222200fsMomentum}.

While figure \ref{fig:OCS2227fsMomentum} gives us some insight into the measurements, we are looking at a multidimensional data set and might want to look at correlations between each measurement. We do this using a pairs plot, or a scatter plot matrix, which is a grid of scatterplots showing the bivariate relationships between all pairs of variables in a multivariate dataset. The pairs plot was first introduced by \citet{Hartigan75}, however we use the more modern and generalized version introduced by \citet{Emerson13} as they provide an open-source implementation in the form of an R package.

Figure \ref{fig:OCS2227fsMomentumPairPlots} uses a pairs plot to showcase the relationship between every pair of variables in our dataset. Due to the $9$-dimensional nature of our dataset, we end up with a $9\times9$ grid of plots with some redundancy. The diagonal repeats the kernel density estimates shown in figure \ref{fig:OCS2227fsMomentum} but they are quite useful as a reference here. Below the diagonal are the scatter plots, however due to the high density of points, contour plots are employed to showcase the same relationship above the diagonal. So only 36 scatter plots are required but this format provides us with greater insight of our dataset, especially in scatter plot regions with a large density of points.

\begin{figure}
  \centering
  \includegraphics[width=\textwidth]{Plots/OCS2227fsMomentumPairsPlot}
  \caption[Pairs plot showing the bivariate relationship between each atom's momentum components measured after Coulomb explosion by a \SI{7}{\fs} laser pulse for the $(2,2,2)$ fragmentation channel.]
  {Pairs plot showing the bivariate relationship between each atom's momentum components measured after Coulomb explosion by a \SI{7}{\fs} laser pulse for the $(2,2,2)$ fragmentation channel. On the diagonal, kernel density estimates (with a Gaussian kernel, see section \ref{sec:kde}) of the momentum component designated by the label at the top of the column and end of the row are given. Below the diagonal, scatter plots show the relationship between the momentum components belonging to that row and column. Above the diagonal, the same relationship is given using a contour plot instead.}
  \label{fig:OCS2227fsMomentumPairPlots}
\end{figure}

For example, the scatter plot in row $7$, column $1$ plots the oxygens's $p_x$ component on the $x$-axis and the sulfur's $p_x$ component on the $y$-axis. The contour plot in row $1$, column $7$ shows the same relationship. We see a negative correlation between oxygen's $p_x$ and sulfur's $p_x$ as predicted in figure \ref{fig:OCS2227fsMomentum}'s caption. Similarly we see negative correlations between oxygen's $p_y$ and sulfur's $p_y$ as well as between oxygen's $p_z$ and sulfur's $p_z$, which makes physical sense due to the two atoms being terminal atoms, so they should fly off in approximately opposite directions following a Coulomb explosion.

Pairs plots for the 30, 60, 100, and \SI{200}{\fs} data sets are included in appendix \ref{appx:supplementaryFigures} as figures \ref{fig:OCS22230fsMomentumPairPlots}--\ref{fig:OCS222200fsMomentumPairPlots}.

\subsection{Discrepancy in the momentum measurements} \label{ssec:momentumDiscrepency}
It is also worth looking at the measurements of the momentum vectors for the three atoms to check if they sum to zero, that is if momentum is conserved. We will look at the momentum measurements for a laser pulse length of \SI{7}{\fs}. We plot a histogram of the momentum discrepancy $\Vert \mathbf{p}_O + \mathbf{p}_C + \mathbf{p}_S \Vert$ exhibited by the measurements in figure \ref{fig:OCS2227fsMomentumDiscrepency}(a). In (b) we also plot a histogram for the total momentum $\Vert \mathbf{p}_O \Vert + \Vert \mathbf{p}_C \Vert + \Vert \mathbf{p}_S \Vert$ for comparison.

\begin{figure}
  \centering
  \includegraphics[width=\textwidth]{Plots/OCS2227fsMomentumError}
  \caption[Discrepancy in the momentum measurements.]
  {(a) Momentum discrepancy and (b) total momentum for momentum vectors measured for the \ch{OCS} $(2,2,2)$ molecule after Coulomb Coulomb by a \SI{7}{\fs} laser pulse. The vertical black line indicates the expected momentum when exploding from the ground-state geometry.}
  \label{fig:OCS2227fsMomentumDiscrepency}
\end{figure}

We see that our the measured momentum vectors sum to zero with a discrepancy of less than \SI{5e-23}{\kilogram\metre\per\second} in the vast majority of cases. This is compared to the total momentum which peaks around \SI{2.5e-21}{\kilogram\metre\per\second}. Thus the discrepancy corresponds to a $<2\%$ error due to momentum not being conserved.

The vertical black line in figure\ref{fig:OCS2227fsMomentumDiscrepency}(b)  indicates the expected momentum when exploding from the ground-state geometry ($r_\mathrm{CO} = \SI{1.15}{\angstrom}, r_\mathrm{CS} = \SI{1.56}{\angstrom}, \theta = \SI{172}{\degree}$), which we sometimes call a $100\%$ Coulombic explosion. In fact the total momentum is distributed about the expected momentum, suggesting that the \ch{OCS} $(2,2,2)$ fragmentation channel is ``highly Coulombic'', or that the Coulomb potential is a good approximation. The agreement is not perfect, however, and this will introduce some error in the reconstructed geometries as we assume a purely Coulombic potential. Significant amounts of initial momentum carried by any of the atomic fragments will also introduce error in the reconstructed geometries.

\subsection{Kernel density estimation} \label{sec:kde}
Kernel density estimates (KDE's) are used to estimate probability distributions in figures \ref{fig:OCS2227fsMomentum} and \ref{fig:OCS2227fsMomentumPairPlots}, and will be used to estimate probability distributions for the atomic positions of reconstructed geometries in chapters \ref{ch:lookupTable} and \ref{ch:optimization}. They are a method of performing nonparametric statistics, that is, of fitting observations to a probability distribution that has no dependency on a parameter \citep[\S 20.2-20.3]{Kendall99}. They serve roughly the same purpose as a histogram, however histograms tend to be non-smooth and their shape depends on both the width of the bins and the ends points of the bins. To solve at least the first two problems, we can use a KDE. They are especially effective for estimating high-dimensional probability distributions where histograms can be very sparse (see also curse of dimensionality, section \ref{ssec:curse}).

For independent and identically distributed univariate samples $x_1, x_2, \dots, x_n$ drawn from an unknown probability distribution $f$, its KDE is
\begin{equation}
\hat{f}_h(x) = \frac{1}{n} \sum_{i=1}^n K_h(x-x_i)
= \frac{1}{nh} \sum_{i=1}^n K\left(\frac{x - x_i}{h}\right)
\end{equation}
where $K$ is the \emph{kernel}, a function with zero mean that integrates to one, and $h > 0$ is the \emph{bandwidth}, a smoothing parameter analogous to the bin width \citep[p. 137]{Scott15}. Many kernel choices exist such as the rectangle or triangle function but the standard normal (or Gaussian) kernel is the most popular. Multivariate KDE's replace the scalar bandwidth parameter $h$ with a symmetric and positive bandwidth matrix $\mathbf{H}$ with a variable number of smoothing parameters \citep{Wand93} however full bandwidth matrices seem to give markedly better performance \citep{Duong03}.

Selection of the bandwidth $h$, as expected, is the most difficult aspect of KDE. Analytical formulae can be derived for the bandwidth $h$ \citep[p. 143]{Scott15} that accurately estimate the unknown density $f$ by minimizing the mean integrated squared error (MISE) however they require knowledge of the unknown density $f$ and so cannot generally be used. Fortunately, when estimating a normal (or Gaussian) density, the bandwidth that minimizes the MISE is given by \citet{Silverman86} as
\begin{equation}
h = \left(\frac{4\hat{\sigma}^5}{3n} \right)^{1/(d+4)}
\approx 1.06 \left(\frac{\sigma^5}{n} \right)^{1/(d+4)}
\end{equation}
where $n$ is the number of observations, $\hat{\sigma}$ is the standard deviation of the samples, and $d$ is the dimensionality of the observations (so $d=1$ for univariate observations and $d=2$ for bivariate). This is termed the \emph{normal distribution approximation} or the ``rule of thumb''. As most of our data, especially the atomic positions of reconstructed geometries, seems to roughly follow a normal distribution, we used the \citet{Silverman86} rule of thumb in computing our kernel bandwidths. It is worth noting that it does not perform well on multimodal data which is why certain estimates in figure \ref{fig:OCS2227fsMomentum} appear over smoothed.

Other methods do exist, and univariate estimators are especially effective as briefly surveyed by \citet{Jones96}, but bandwidth selection for multivariate kernel density estimates is quite difficult for non-Gaussian densities \citep[\S 6.5.2]{Scott15}.

\section{Computationally simulating a Coulomb explosion} \label{sec:simulating}
To simulate a Coulomb explosion of a molecule containing $n$ atoms, we will make some simplifying assumptions to arrive at the simplest possible simulation that will allow us to investigate the problem of reconstructing geometries. We will assume that the motion of the ions are governed only by their mutual Coulomb repulsion, so that the chemical bonds are broken instantaneously at $t = 0$ and have no effect on the trajectories of the ions, and that neutral fragments do not interact with any other fragment. We model each atom as a point particle with a fixed electric charge assigned at $t = 0$ so no charge redistribution can occur. We assume that the atoms each begin at rest at $t = 0$ and that their initial positions are determined by the molecule's equilibrium or ground-state geometry. Of course, these assumptions force us to ignore the rearrangement of the atoms under the influence of the laser pulse's intense electromagnetic field, and any initial momentum imparted on the atoms by this interaction.

Under such assumptions, we can solve the classical equations of motion for each ion right after the explosion. We choose to use Hamiltonian mechanics to acquire a system of $6n$ first-order ordinary differential equations (ODE's) which may be easily solved by numerical methods such as the ubiquitous fourth-order Runge-Kutta. If Lagrangian mechanics is used, then the resulting second-order must be recast as a system of first-order ODE's as numerical algorithms are developed to solve systems of first-order ODE's. The Hamiltonian of the molecular system is
\begin{equation}
\mathcal{H}(\mathbf{r}_i, \mathbf{p}_i, t) = \sum_{i=1}^n \frac{\mathbf{p}_i^2}{2m_i} + \frac{1}{4\pi\epsilon_0}\sum_{\substack{\lbrace i,j\rbrace\\ i \ne j}} \frac{q_iq_j}{|\mathbf{r}_i-\mathbf{r}_j|}
\end{equation}
where $i,j \in \lbrace 1,2,\dots, n \rbrace$ and so the second summation is over all $i,j$ pairs where $i \ne j$. Calculating Hamilton's equations for the system, we get
\begin{subequations}
  \begin{align}
  \frac{d\mathbf{r}_i}{dt} &= \frac{\partial \mathcal{H}}{\partial \mathbf{p}_i} = \frac{\mathbf{p}_i}{m_i} \\
  \frac{d\mathbf{p}_i}{dt} &= -\frac{\partial \mathcal{H}}{\partial \mathbf{r}_i} = -\frac{1}{4\pi\epsilon_0}\sum_{j, \; j \ne i} \frac{\mathbf{r}_i - \mathbf{r}_j}{|\mathbf{r}_i - \mathbf{r}_j|^3}
  \end{align}
\end{subequations}
where $i$ is held fixed over the second summation. With appropriate initial conditions this system of $6n$ scalar first-order ordinary differential equations may be easily solved using, for example, the classical fourth-order Runge-Kutta method for numerically solving ordinary differential equations. The atoms are assumed to be at rest so that $\mathbf{p}_i(t=0) = 0$, while the initial positions, $\mathbf{r}_i(t=0)$, are chosen to correspond to the molecular geometry.

One way to think of the problem being tackled in this thesis is: which initial geometry $\mathbf{r}_i(t=0) = 0$ results in the momentum values measured at the detector? The atoms are far enough apart after just a few nanoseconds that by the time they arrive at the detector, they feel almost no forces due to each other and their momenta attain asymptotic values which we can denote $\mathbf{p}_i(t\rightarrow\infty)$.

For some perspective, \citet{Slater15} discuss the computational simulation of ion trajectories for larger systems such as the the 3,5-dibromo-3',5'-difluoro-4'-cyanobiphenyl (\ch{DBrDFCNBph}) molecule, and who interestingly employ a pixel-imaging mass-spectrometry camera for position detection.

\section{Describing geometries and momenta} \label{sec:conventions}
While tackling the problem of geometry reconstruction, it will be crucial to choose a convention for describing the geometries and momentum vectors especially so that geometries and vector arrangements can be compared with ease. Even more importantly, it provides us with an opportunity to reduce the dimensionality of the problem from $3N$ to $3N-6$ for a molecular system with $N$ atoms. This stems from the fact that we only need to describe the relative position of each atom, not its absolute position.

For example, a triatomic molecule can be described by two bond lengths $r_{12}, r_{23}$ and a bond angle $\theta$, or even three bond lengths ($r_{12}, r_{13}, r_{23}$) rather than three position vectors $(\mathbf{r}_1, \mathbf{r}_2, \mathbf{r}_3)$.

For larger molecules, dihedral angles are required to describe the geometry in addition to bond lengths and angles. Such coordinates are called \emph{internal coordinates} and multiple possible descriptions may exist \citep{Peng96} although it is unclear how significantly the choice of coordinate system will come into effect for molecules containing several atoms. The Z-matrix, a tool from computational chemistry, may be used to store and convert between them and Cartesian coordinates using the Natural Extension Reference Frame (NERF) algorithm \citep{Parsons05}. However, concerns regarding the Z-matrix have been discussed \citep{Baker91,Baker96} and further investigation is required 

Before attempting to compare simulated momentum vectors with experimentally measured data, the momentum vector measurements must be converted from the laboratory frame to the center-of-momentum (COM) frame by removing any center-of-mass motion of the molecular system (see the \texttt{removeCOMMotion.m} code listing in appendix \ref{appx:code}).

The exact same molecule can produce different momentum vectors after a Coulomb explosion depending on its initial orientation with respect to the detector. We must use a momentum convention to ensure a one-to-one mapping between geometries and measured momentum vectors as we are strictly interested in the molecular structure, and not the molecule's orientation.

For triatomic molecules, the momentum vectors can be rotated into a plane (see the \texttt{rotateMomentum.m} code listing in appendix \ref{appx:code}). We further restrict the central atom's momentum vector to lie along the $+x$-axis and that the second terminal atom sits in the $+x$ half plane. Thus we are only left with five nonzero momentum components as $\mathbf{p}_1 = (p_{1x}, p_{1y}, 0)$, $\mathbf{p}_2 = (p_{2x}, 0, 0)$, and $\mathbf{p}_3 = (p_{3x}, p_{3y}, 0)$. With conservation of momentum, $\mathbf{p}_3 = -(\mathbf{p}_1 + \mathbf{p}_2)$, and so we just need three momentum components before we can determine all the momentum vectors.

For larger molecules, the momentum vectors cannot always be rotated into a plane although we can still employ a very similar procedure. After the first three vectors are rotated into a plane, the remaining vectors may point out of the plane.
