\chapter{A brief history of molecules}\label{ch:history}

\vspace{-1.5 em}
\minitoc\hrule
\vspace{1.5 em}

It is easy to get stuck in the ivory tower and get lost within the trees of science, losing sight of the big picture of where your research leads.

\subsection{Molecules since antiquity}

\section{If molecules exist, what do they look like?}
\subsection{Breakthroughs}
% \subsection{Why CEI?}

\section{Molecular movies}
% \subsection{The first movies}

The idea that Coulomb explosion imaging could be used to produce so-called molecular movies has permeated the literature ever since the technique's emergence.

CEI was first performed by \citet{Vager89} whereby the Coulomb explosion was initiated by passing the molecule through a thin carbon film at high velocities.

There exist other methods of initiating CEI, among them thin foils, free-electron lasers, highly-charged ion impact, single X-ray photons from a synchrotron source.

\index{Classical imaging formula}
\index{Coulomb explosion imaging!Classical imaging formula}
Surprisingly, an attempt was made to arrive at an analytical solution for calculating geometries from measured momentum data. \citet{Nagaya04} \marginpar{They seem to have worked hard to find an analytical solution but their unsaid conclusion seems to be that it is an intractable problem and their group went silent on this problem.} are able to derive so-called classical imaging formulas giving the position wavefunction squared for the Coulomb explosion of a diatomic molecule and a linear triatomic molecule (the cases of symmetric and asymmetric Coulomb explosion are treated).

\index{Coulomb explosion imaging!History}
\citet{Legare05structure,Legare05dynamics} was the first to use Coulomb explosion imaging and report molecular structures. To obtain structures, they assume the explosion proceeds under a purely Coulombic potential and use optimization methods to make guesses at the structure that most accurately reproduces the observed data consistent with minimizing a least-squares objective function. Unfortunately they provide very minimal information regarding their methods and there is a complete lack of discussion acknowledging the shortcomings of this method.\marginpar{The main shortcomings being degenerate solutions and the fact that they employ convex optimization methods to a problem that is not convex. It is not clear if they even knew about these issues.} Using 8 fs laser pulses they report on the structure of D2O and SO2 \citep{Legare05structure}. They also claim to have imaged vibrating D2+ and dissociating SO22+ and SO23+ however they provide no more than a couple of dissociation frames and infer the transient D2+ bond length from kinetic energy release ratios as a function of pump-probe time delay \citep{Legare05dynamics}.

\citet{Gagnon08} reported the reconstruction of dichloromethane (CH2Cl2) using a home-made \marginpar{There is nothing wrong with writing your own code here but nonconvex optimization algorithms are tricky to get right and the reliance should be on professional code.} stochastic-based simulated annealing algorithm that globally optimizes the molecular spatial configuration. They discuss uncertainties but are only able to obtain the structure in five cases. \footnote{See \citet{Bocharova11} for a discussion of optimization algorithms.}

\index{Lookup table}
The best effort so far has probably been the one by \citet{Kunitski15} in which they use a lookup table approach to image the Efimov state of the helium trimer.

Molecular movies are of course not only of interest in physics and chemistry as a means of probing fundamental processes, but also in the biological sciences where molecular structure play a crucial role in determining the function of biomolecules such as proteins. However, the molecules of interest there are much too large to be studied by any of the previous techniques. Thus molecular movies in the biological sciences tend to be annotated computer simulations amalgamated from multiple studies. That said, they are very impressive pieces of work.

A particularly impressive movie by \citet{Cheung12} showcases the process of RNA polymerase transcription and goes on for over six minutes.

