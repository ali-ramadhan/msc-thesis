\chapter{A brief history of molecules}\label{ch:history}

\vspace{-1.5 em}
\minitoc\hrule
\vspace{1.5 em}

What is Coulomb explosion imaging? A technical definition could be given, however, it is simply a decades-old technique in a long line of techniques stretching back centuries, all meant to answer a question posed by the ancient Greeks and Indians---what are the building blocks of the universe, and how do they behave?

It is easy to get stuck in the ivory tower and get lost within the trees of science, losing sight of the big picture of where your research leads.

\subsection{Molecules since antiquity}

\section{If molecules exist, what do they look like?}
\subsection{Breakthroughs}
% \subsection{Why CEI?}

\section{Molecular movies}
% \subsection{The first movies}
Molecular movies are of course not only of interest in physics and chemistry as a means of probing fundamental processes, but also in the biological sciences where molecular structure play a crucial role in determining the function of biomolecules such as proteins. However, the molecules of interest there are much too large to be studied by any of the previous techniques. Thus molecular movies in the biological sciences tend to be annotated computer simulations amalgamated from multiple studies. That said, they are very impressive pieces of work.

A particularly impressive movie by \citet{Cheung12} showcases the process of RNA polymerase transcription and goes on for over six minutes.

