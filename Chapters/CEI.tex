\chapter{Coulomb explosion imaging}\label{ch:CEI}

\section{Experimental scheme}
In pump-probe Coulomb explosion imaging (CEI) one ultrashort laser pulse is split into two pulses through the use an asymmetric beamsplitter. One of the pulses, the pump pulse, is usually much weaker than the other, the probe pulse. A time delay $\tau$ between the pulses is then created such that the pump pulse goes first and the probe pulse second. The job of the pump pulse will be to initiate some change in the molecule. One example could include an isomerization of the molecule. Thus the pump pulse ``pumps'' the molecule into some excited state. The job of the powerful probe pulse is to engulf the molecule in an intense enough laser field such that multiple electrons are stripped off of it. The molecule's individual atoms are left in a highly-charged state and begin to behave as individual point charges in a purely Coulombic potential. The entire process occurs in the presence of a constant electric field and so the positively-charged ions all accelerate upwards towards a time- and position-sensitive detector. Thus the probe pulse allows for the ``probing'' of the excited state.

Suppose $U$ is a simply connected open subset of the complex plane, and $a_1,\dots,a_n$ are finitely many points of $U$ and $f$ is a function which is defined and holomorphic $U \; \textbackslash \; \{ a_1,\dots,a_n \}$. If $\gamma$ is a closed rectifiable curve in $U$ which does not meet any of the $a_k$,

\begin{equation}
\oint_\gamma f(z)\, dz = 2\pi i \sum_{k=1}^n \operatorname{I}(\gamma, a_k)  \operatorname{Res}( f, a_k ).
\end{equation}

\subsection{Femtosecond Multiple Pulse Length Spectroscopy}
\subsection{Pump-probe Coulomb explosion imaging}

\section{Experimental details}

\section{History and accomplishments}
CEI was first performed by \citet{Vager89} whereby the Coulomb explosion was initiated by passing the molecule through a thin carbon film at high velocities.

\section{Other ways of initiating Coulomb explosions}
There exist other methods of initiating CEI, among them thin foils, free-electron lasers, highly-charged ion impact, single X-ray photons from a synchrotron source.

\section{Time and position measurement}

\section{Calculating atomic fragment momenta}

\section{Uncertainty in momentum measurements}