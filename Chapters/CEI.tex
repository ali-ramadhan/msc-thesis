\chapter{Coulomb explosion imaging}\label{ch:CEI}

\vspace{-1.5 em}
\begin{addmargin}[-0.5cm]{0cm}
  \minitoc
\end{addmargin}
\hrule
\vspace{1.5 em}

% The scope of this imaging method is to make measurements of the geometries of single molecules on a timescale faster than that of molecular motion ($10^{-15}$ seconds). Ultimately these images can be recorded in sequence to image the dynamics of a molecule. It works well for small molecules in the gas phase, a regime in which no other method has been viable.

% The method involves shooting a molecule in a vacuum chamber with two laser pulses, a pump pulse followed by a probe pulse after some time delay $\tau \ge 0$. The pump causes the molecule to undergo some change, then the more intense probe pulse strips off enough electrons from the molecule such that the molecule's individual atoms separate and repel each other ``explosively'' due to Coulomb repulsion. The further apart the atoms are repelled, the weaker the repulsion and eventually every atom reaches an asymptotic state of constant velocity. This all takes place in a constant electric field that accelerates all the atoms towards a time and position-sensitive detector. The detector can tell us how much time each atom took to reach it, and the position where the atom hit it. From this information, the atom’s (asymptotic) momentum can be calculated and then provided we know how to simulate the experiment in reverse, we can determine the initial geometry of the molecule just before it was hit by the pump pulse. By picking different values of $\tau$ and repeating the experiment many times for each $\tau$ we get a ``frame'' of the molecule's geometry at each $\tau$ , which may be placed in sequence to form a ``molecular movie'' of the change induced by the pump pulse.

% We are only interested in the molecular geometry, i.e. the relative positions of the atoms, not the absolute position of every atom. This reduces the dimensionality of our problem from $3n$ to $3n-6$ in both the geometry $\mathbf{g}$ and momentum vector $\mathbf{p}$. The geometry ``vector'' $\mathbf{g} \in \mathbb{R}^{3n-6}$ contains the molecule's bond lengths and bond angles, and the momentum ``vector'' $\mathbf{p} \in \mathbb{R}^{3n-6}$is a concatenation of momentum values in a specific convention. They both do not transform like vectors.

% While explosions proceed in a deterministic fashion, that is structures map bijectively to momentum measurements, the converse is not true. Two very different structures may produce the same momentum measurements. To make matters worse, there is no analytic solution to the problem and as the molecule grows, the problem of finding its structure becomes increasingly high dimensional. To combat this problem we will require the use of various mathematical and statistical methods. I first discuss some results from the theory of inverse problems to shed some general insight on these problems. I then follow with a discussion of optimization methods which may be used to tackle the problem for very small molecules. However, for full imaging of larger polyatomic molecules with an analysis of measurement error, Bayesian inference using Markov chain Monte Carlo methods is the way to go, which I discuss in the last chapter of this part.

% To find the initial geometry $\mathbf{g}_0$ that produced the measured asymptotic momentum $\mathbf{p}_\infty$, we make use of the fact that simulating the Coulomb explosion forward in time is easy. Let $\mathbf{p}(\mathbf{g}) : \mathbb{R}^{3n-6} \rightarrow \mathbb{R}^{3n-6}$ map an initial geometry $\mathbf{g}$ onto the asymptotic momentum vector $\mathbf{p}_\infty$ such a geometry produces upon Coulomb explosion. It simulates the explosion forwards in time by solving a set of $6n-12$ coupled first-order ODEs. Then casting this as an optimization problem, we seek the geometry $\mathbf{g}$ that minimizes the objective function $\left\| \mathbf{p}(\mathbf{g}) - \mathbf{p}_\infty \right\|_2^2$.

\section{What's the deal with Coulomb explosion imaging?}
Coulomb explosion imaging (CEI) is a general technique for studying the structure and ultrafast dynamics of small molecules in the gas phase, essentially by ionizing the molecule to induce fragmentation after which the positively-charged fragments repel each other in a \emph{Coulomb explosion}\footnotemark~ and the momentum vector of each fragment is measured. In principle, many claim that it is possible to retreive the molecular structure with just the momentum vectors, and we will a general method for doing this in the following chapters, however we will find out that this is a problem fraught with uncertainty to the point that geometry reconstructions cannot be trusted to provide us with insights into a molecule's structure. The momentum vectors may themselves be studied to infer molecular dynamics and changes in molecular structure.

\footnotetext{It is worth mentioning that the concept of a Coulomb explosion is indepdendent of CEI and refers to a cluster of many atoms repelling each other under their mutual Coulomb repulsion following ionization, for example by the intense electromagnetic field of a short laser pulse. Interestingly, a Coulomb explosion seems to be the mechanism responsible for the explosive reaction of alkali metals, such as sodium or potassium, with water, finally explaining the chemistry behind the classic high school experiment \citep{Mason15}.}

This does not sound like a technique of interest in the 21st century---the structure of virtually all small gas molecules is well known, what is left for CEI to tell us about? Producing molecular movies of ultrafast molecular dynamics is a main goal with imaging proton migration being a prototypical example. Once these ultrafast chemical processes are imaged and understood, it may be possible to control them. It can also tell us about non-classical molecular structures that elude other methods, such as that of \ch{C2H3+} and the Efimov state of the helium trimer (\ch{4He3}), and can serve more specialized purposes such as identifying the absolute geometry of molecular isomers.

Another reason for CEI's popularity may be that it promises direct and intuitive retreivals of molecular structure. Perhaps such physical reconstructions that can be understood by inspection satisfy our curiousity more so than complicated x-ray diffraction patterns. Certainly at least one author became interested with the prospect of \emph{seeing molecules} using lasers and explosions. CEI directly answers the question of ``what are we made of and what does it look like?'' first posed by the ancient Greeks and Indians, most famously by Democritus. The idea of the four classical elements: earth, air, fire, and water was proposed by many cultures to explain natural phenomena and the complexity of matter in terms of simpler entitites with some explanations being tied to atomism, the idea that matter was made of tiny, indivisible entities, such as Plato's association of the Platonic solids with the classical elements (see figure \ref{fig:platonicSolids}). While the ideas evolved over time, it was not until the 1600's that the theory was subject to experimental verification and eventually completely disproved. Even today, the classical elements are commonly employed and referred to in popular media. Such is the power of simple explanations.

\begin{figure}
  \centering
  \includegraphics[width=\textwidth]{gfx/PlatonicSolids}
  \caption[The classical elements associated with the five Platonic solids.]
  {The classical elements associated with the five Platonic solids. Clockwise from the top left: the octahedron with air, tetrahedron with fire, dodecahedron with the universe, the isocehedron with water, and the cube with earth. Figure from \citet[Book 2, p. 53]{Kepler1619}. English translation available \citep{Kepler97}.}
  \label{fig:platonicSolids}
\end{figure}

\section{Experimental outline} \label{sec:CEIphysics}
CEI must be performed under high-vacuum conditions and so the molecule must be ionized and detected within a position-sensitive time-of-flight (PSToF) spectrometer that lies inside a vacuum chamber. Figure \ref{fig:spectrometer} shows a very basic schematic of a CEI PSToF spectrometer used by \citet{Ramadhan16}. The molecules of interest may be introduced simply as an effusive gas jet but for laser CEI, a well-collimated supersonic jet of cold molecules is highly desired to increase the resolution with which the time-of-flight and position measurements may be made \citep{Dorner00} as well as prepare molecules with a known velocity distribution.

\begin{figure}
  \centering
  \includegraphics[width=\textwidth]{gfx/SpectrometerLaser}
  \caption[Basic chematic of a position Ssnsitive time-of-flight spectrometer.]
  {Basic chematic of a position sensitive time-of-flight spectrometer. A cross-section of the rings are shown, which set up the constant electric field by virtue of being biased to different voltages through the use of a resistor chain. The size of the example triatomic molecule is \emph{greatly} exaggerated for illustrative purposes. Note that electrons will be accelerated downwards, and may be detected using a separate second detector. This figure was adapted from an earlier version meant to accompany \citet{Ramadhan16}.}
  \label{fig:spectrometer}
\end{figure}

In pump-probe CEI the ultrashort laser pulse is split into two pulses through the use a beam splitter. The first is called the \emph{pump pulse} and is usually weaker than the second, the \emph{probe pulse}. A time delay $\tau$ between the pulses is created by exposing the probe pulse to a longer optical path length such that the pump arrives at the molecule at time $t=0$ followed by the probe pulse at time $t=\tau$. The job of the pump pulse is to initiate some change in the molecule, ideally a change that we are interested in imaging. One important example would be an isomerization of the molecule \citep{Ibrahim14,Liekhus-Schmaltz15}. Thus the pump pulse \emph{pumps} the molecule into some excited state.

The job of the more powerful probe pulse is to engulf the molecule in an intense enough electric field such that multiple electrons ($\ge2$) are stripped off of it. The molecule is left in a highly charged and unstable state where the chemical bonds between the individual atoms cannot hold it together any longer, and the molecule dissociates. It may dissociate completely into its constituent atoms or dissociate partially into a mixture of atoms and molecular fragments. The fragments are left in a highly-charged state (although some neutral fragments may be produced) and begin to behave as individual point charges in a weak Coulombic potential. The entire process occurs in the presence of a constant electric field and so the positively-charged ions accelerate upwards towards the PSToF detector, which allows for the measurement of their momentum vectors (section \ref{sec:measurement}). Thus the probe pulse allows for the \emph{probing} of the excited state.

The time delay $\tau$ between the pulses may be varied to image the dynamics induced by the pump pulse at various times. Knowledge of the molecular structure at each time delay allows for the production of a series of frames showcasing the dynamics, or a \emph{molecular movie}.

An alternative to pump-probe CEI uses a single laser pulse to both pump and probe the molecule. In this case, the laser pulse length is varied (not the time delay between two laser pulses) acting like the shutter speed of a camera. This allows for the imaging of ionization dynamics, and the variable pulse length allows for some control over the ionizaton process.  \citet{Karimi13} provide a review of this technique, termed Femtosecond Multiple Pulse Length Spectroscopy (FEMPULS).

For details on the physical principles, \citet{Posthumus04} provides a detailed review on the dynamics of small molecules in intense laser fields.

\section{Molecular geometries seen using CEI}
The original CEI experiment is usually traced back to \citet{Vager89} in which the Coulomb explosion is initiated by passing a molecular beam through a thin foil. This may be because it was the first work suggesting that full molecular structures may be recovered by measuring the velocity (or momentum) vectors of the atomic fragments, and even reported on a non-classical molecular structure. However, previous works utilizing CEI do exist, even some significant works that report on molecular structures \citep{Kanter79}.

Ultrashort laser pulses\footnotemark as a means of inducing Coulomb explosions made their entrance in the 1980's where they were utilized to infer molecular dynamics using covariance mapping \citep{Frasinski89}. Highly charged ion impact is another method of inducing a Coulomb explosion, and was first done in the 1990's in parallel with the development of more sophisticated coincidence mapping techniques. Since then, the laser has emerged as the more popular tool and has further developed the coincidence mapping technique. There do exist other methods of inducing Coulomb explosions, for example, single photons from a synchrotron source utilizing the Auger effect, x-ray pulses from a free-electron laser source, or electron collision.

\footnotetext{In 1987, ultrashort would be referring to \SI{0.6}{\pico\s} laser pulses \citep{Frasinski87}.}

In this section we will trace the history of CEI back to the 1970's where it started with foil-induced fragmentation. We will then follow it's development to the present day where ultrashort laser pulses are the most popular means of performing CEI. Throughout we will focus solely on the achievements of CEI in determining molecular structures, and in creating molecular movies using these recovered structures.\footnotemark

\footnotetext{Much of the molecular dynamics are inferred in CEI from studying the distribution of the fragment momentum vectors (\eg~ through the use of Newton and Dalitz plots) and the distribution of kinetic energy carried away by each fragment. We will be focusing on the original aim of CEI, that is, to measure molecular structures.}

Interestingly, the first-ever mention of the term ``Coulomb explosion'' in the published literature comes from an unrelated study of the fine structure of singly ionized helium by \citet{Novick55}. They measured the energy difference of the $2 \, ^2 S_{1/2}$ and $2 \, ^2 P_{1/2}$ states of ionized helium as a sensitive test of quantum electrodynamics. Coulomb explosion (or space charge explosion) was the dominant ion removal mechanism which they accounted for in modeling the quenching rate\footnotemark of metastable $2 \, ^2 S_{1/2}$ ions by radio frequency radiation to describe the observed resonance lineshapes (spending two appendices on it).

\footnotetext{The term was more popular in decades past but simply means the extinction rate or loss rate of metastable ions.}

\subsection{Foil-induced dissociation}
CEI was first performed by passing a molecular beam containing the molecular ion of interest from a storage ring through a thin atomic film. While in the solid film, the probability for Coulomb scattering of the individual atomic nuclei is small due to their small size and consequently small interaction cross-section. The electrons will be scattered to very wide angles due to their interaction with the many electron clouds in the film. This process rapidly ionizes the molecule, typically within the first few atomic layers, or the first femtosecond. This time scale is faster than the characteristic times scale for molecular vibration ($>\SI{e-14}{\s}$) and rotation ($>\SI{e-12}{\s}$). The now highly ionized molecule exits the foil and rapidly breaks up into its consituent atomic ion fragments which repel eachother under their mutual Coulomb repulsion in what is termed a ``Coulomb explosion'' \citep{Vager89}. Figure \ref{fig:foilExperiment} shows a schematic of such an experiment. This specific experiment set out to measure the absolute configuration of atoms in a chiral molecule in the gas phase, which remains challenging. Early foil-induced CEI experiments can be described by this schematic except for the fact that they did not employ a mass selector and used a more primitive but still position-sensitive detector.  The atomic fragment trajectories (dashed lines) assume that no rearrangement of the atoms occurs and that the system evolves under a Coulomb potential.

\begin{figure}
  \centering
  \includegraphics[width=\textwidth]{gfx/FoilExperiment}
  \caption[Schematic of a foil-induced Coulomb explosion imaging experiment.]
  {Schematic of a foil-induced Coulomb explosion imaging experiment. From \citet{Herwig13}. Reprinted with permission from AAAS.}
  \label{fig:foilExperiment}
\end{figure}

The premise behind foil-induced CEI is that during this explosion, the atoms simply repel each other and do not rearrange, thus preserving the angles between them from the time they exit the foil to the time they are detected at a position and time-sensitive detector. As the potential energy of each pair of fragments $i,j$ is coverted to kinetic energy according to
\begin{equation} \label{eq:foilCEI}
\frac{4q_i q_j}{|\mathbf{r}_i - \mathbf{r}_j|} = \frac{\mu|\mathbf{V}_i - \mathbf{V}_j|^2}{2}
\end{equation}
it suggests that measurement of the asymptotic vector velocities completely defines the initial geometry of the molecule. Here $q_i$, $\mathbf{r}_i$, and $\mathbf{V}_i$ are the charge, position vector, and velocity vector of the atomic fragment $i$ while $\mu$ is the reduced mass of the two-body system of fragments $i,j$. \citet{Vager89} argues that the density of the measured $\mathbf{r}_i$ vectors is an experimental measurement of the square of the three-dimensional nuclear ground-state wave function, and would additionally describe all possible correlations between the molecule's constituent nuclei.

\begin{figure}
  \centering
  \includegraphics[width=\textwidth]{gfx/HydrogenTrimerReconstruction}
  \caption[Reconstructions of exploded \ch{H3+} following foil-induced dissociation.]
  {Reconstructions of exploded \ch{H3+} following foil-induced dissociation. On the top row are two photographs of exploded \ch{H3+} recorded on photographic emulsion at a tilt angle of $30\degree$. The bottom row shows a few reconstructions (normally projected) made by inspection from the photographs. The authors analyzed $350$ such photographs and concluded that \ch{H3+} mainly exhibits an equalateral triangle geometry. A distribution of angles for the equilateral geometries is given as well. From \citet{Gaillard78}. Reprinted with permission from APS.}
  \label{fig:hydrogenTrimer}
\end{figure}

The earliest example of a molecular geometry recovered using CEI is reported by \citet{Gaillard78}. They used the foil-induced Coulomb explosion to image the structure of the \ch{H3+} molecular ion, showing that it mainly exhibits an equalaterial triangular shape using separate pieces of evidence from three different experiments.\footnotemark ~Figure \ref{fig:hydrogenTrimer} gives a few examples of the geometries they recovered.

\footnotetext{It is interesting to note that the experiment was repeated by three separate teams, then reported on coherently in one manuscript. Each team, having access to different equipment, produced separate pieces of data. It was the team in Rehovot, Israel that recorded the projections of the exploded ions while the others measured energy spectra and angular distributions.}

It is unclear how the idea for such an imaging experiment came to be, however it is worth noting that \citet{Gaillard78} and co-authors have been studying the effects of molecular beams passing through thin foils for quite some time, mainly at Argonne National Laboratory. See for example their studies of wake potentials generated behind charged particles as they pass through a solid \citep{Gemmell75, Vager76PRL} and their study of the dissociation of fast \ch{HeH+} ions traversing thin foils \citep{Vager76PRA}. It seems quite reasonable that studying the dissociation of small molecules and the angular distribution of the atomic fragments would inspire researchers to attempt to infer molecular structures using this data.

A better known CEI experiment was performed by \citet{Vager89} almost a decade later employing a $\sim$\SI{30}{\angstrom} carbon film. Their work was motivated by the opportunity of imaging non-classical molecular structures that more established methods were incapble of imaging. They were also the first to suggest that measuring the velocity (or momentum) vectors of each fragment would be provide all the information required to describe the molecule's structure. Figure \ref{fig:C2H3geometry} shows one reconstruction for the \ch{C2H3+} molecule based on the measured velocity vectors. Assuming equation \eqref{eq:foilCEI} holds, it shows an inference of the molecule's geometry based on the argument that is an experimental measurement of the square of the nuclear wave function, and is a convincingly pretty one at that.

\begin{figure}
  \centering
  \includegraphics[width=\textwidth]{gfx/VagerPseudoGeometry}
  \caption[Reconstruction of \ch{C2H3+} following foil-induced dissociation.]
  {Reconstruction of \ch{C2H3+} following foil-induced dissociation. The densities of the fragment ions are plotted in a coordinate system defined by the final velocities of each particle (relative to the mean carbon-ion velocity). The carbon ion densities were reduced by a factor of $5$ for display purposes. From \citet{Vager89}. Reprinted with permission from AAAS.}
  \label{fig:C2H3geometry}
\end{figure}

However, they do not perform any geometry reconstruction and report their fragment ion densities in a coordinate system defined by the asymptotic velocity of each particle. Considering that they provide equation \eqref{eq:foilCEI} for converting from velocity vector measurements to position vectors, it seems rather unusual for them to report a distribution in velocity space as opposed to position space.

The promise of recovering molecular geometries in this simple fashion seems quite empty now after glancing at the published literature in the few decades since. One reason for this include the difficulty and inconvinience of preparing the molecular beam required, which may not be possible for every molecule to be imaged. Another reason is that the method assumes that no molecular rearrangement occurs after the molecule passes through the thin film, which requires the complete dissociation of the molecule. This is a crucial issue no matter the ionization method used, however a thin film may not be very effective at inducing complete dissociation in many molecules of interest, while an ultrashort laser pulse can be more effective.

Foil-induced CEI has found some uses and produced some interesting work in 
recent years such as the imaging of the rovibrational wave functions of \ch{H2-} and \ch{D2-} by studying the kinetic energy relased by the fragments \citep{Jordon-Thaden11, Herwig13PRA}, and imaging the absolute configuration of chiral molecules in the gas phase by studying the measured velocity vectors of the fragments with Newton plots \citep{Herwig13}. \citet{Vager01} also provides a review of some experiments. However, for the purposes of studying molecular structures and dynamics, the ultrafast laser shortly thereafter became the tool of choice for CEI. The reason for the scarcity of foil-induced CEI experiments in the published literature is mainly due to its limitations which also include the scarcity of experimental storage ring facilities, and the requirement the the molecule must be prepared as a molecular ion beam. The latter may prohibit the study of many molecules that cannot be prepared as such or which change molecular structure upon excitation away from the neutral ground state.

Of course, there are some assumptions that must hold for a complete and accurate recovery of the initial geometry. However, just by inspection of the schematic in figure \ref{fig:foilExperiment} we can see that no rearrangement of the atoms must occur and that the molecular system must evolve on a purely Coulombic potential, which requires the rapid stripping of many electrons off the atom. Thus CEI becomes increasingly difficult to perform with larger molecules so it is best used to study smaller molecules. However, it is precisely these small molecules in the gas phase that need to be studied using CEI as they cannot be probed using other more established methods. Moreover, many smaller molecules may exhibit non-Coulombic behaviour unless placed into a highly charged state, which may be impossible depending on the apparatus in use.

\subsection{Imaging with ultrashort laser pulses}\label{sec:laserCEI}
Due to the limitations of foil-induced dissociation for CEI, the ultrashort laser emerged as a powerful tabletop solution for rapidly ionizing molecules. This occured in the early 1990's following the development of the first broad-bandwidth solid-state Ti:sapphire laser by \citet{Moulton86} and the first demonstration of a mode-locked Ti:sapphire laser producing femtosecond laser pulses by \citet{Spence91}, as well as the introduction of a chirped-pulse amplification scheme by \citet{Strickland85} that allowed for the generation of ultrahigh peak power laser pulses \citep{Maine88}.

\subsubsection*{Attempt at an analytical solution}
\index{Classical imaging formula}
\index{Coulomb explosion imaging!Classical imaging formula}
Before taking a tour of the molecular geometries recovered using laser-induced CEI, it is worth mentioning that \citet{Nagaya04} have attempted to arrive at an analytical solution for calculating molecular geometries from the measured momentum vectors. They were able to derive \emph{classical imaging formulae} giving the image of the squared vibrational wavefunction inverted from the momentum distribution of the atomic ions for the Coulomb explosion of a diatomic molecule, a linear symmetric triatomic molecule, and a linear asymmetric triatomic molecule. They are able to derive similarly simple formulae for the diatomic as well as the linear, symmetric, triatomic molecule, but the more general case of the asymmetric, linear, triatomic molecule proves much more formidable.\footnotemark~ The bulk of their article focuses on that case, deriving a three-dimensional classical imaging formula in terms of Jacobi and hyperspherical coordinates then reducing it to two dimensions. An extension to three dimensions for bent triatomic molecules is promised but could not be found in the published literature.

As an example, their formula for the Coulomb explosion of a diatomic molecule $AB$ is given as
\begin{equation}
|\Psi_\mathrm{image}(R_I)|^2 = S(p) \frac{1}{P_\mathrm{ion}(R_I)} \sqrt{\frac{\mu q_A q_B}{8\pi\epsilon_0 R_I^3}}
\end{equation}
where $S(p)$ is the momentum distribution measured in the asymptotic region (when the atomic fragments are far apart and barely interact), $P_\mathrm{ion}(R_I) = |T_\mathrm{ion}(R_I)|^2$ is the ionizatin probability, $\mu$ is the reduced mass of the diatomic molecule, $q_A$ and $q_B$ are the electric charges on atoms $A$ and $B$ respectively, and $R_I = \mu q_A q_B/2\pi\epsilon_0 p^2$ is described elswhere to be the $R$-coordinate  corresponding to the total energy $E$ of the exploding fragments determined by conservation of energy, $E = E_0 + q^2/R_I$ \citep{Chelkowski02}.

They proceed to compare their ``classical'' reconstructions for the vibrational wavefunction of various linear helium trimer systems (\ch{He3^{3+}} and \ch{He3^{6+}}) to the predictions of the quantum theory, noting small discrepencies. It seems that similar formulae were derived and actually used in previous studies \citep{Bandrauk01, Chelkowski02}, however, more recent studies do not seem to employ these classical imaging formulae.

\footnotetext{While a highly commendable effort, their unsaid conclusion seems to be that this is an intractable problem as their research group seems to have gone silent on this problem and futue studies do not refer back to \citet{Nagaya04} except when discussing the difficulty of the problem.}

\subsubsection*{Experimental reconstructions}
\citet{Legare05structure,Legare05dynamics} were the first to use ultrashort laser pulses (\SI{8}{\fs}) and CEI to report on molecular structures and dynamics. Figure \ref{fig:SO2-232structure} shows a reconstruction of \ch{SO2} using the \ch{SO2^7+ $\rightarrow$ O^2+ + S^3+ + O^2+} charge state.

\begin{figure}
  \centering
  \includegraphics[width=\textwidth]{gfx/LegareSO2-232Structure}
  \caption
  [Molecular structure of \ch{SO2} for the \ch{SO2^7+} charge state.]
  {(a) Molecular structure of \ch{SO2} using the \ch{SO2^7+} charge state (\ch{SO2^7+ $\rightarrow$ O^2+ + S^3+ + O^2+}). The center of mass is at $x=0$, $y=0$, and the $y$-axis is the bisector of the angle. (b) Radial distribution and (c) angular distribution of the reconstructed geometries with the dotted lines showing the expected distributions for the $\nu=0$ stationary state structure of \ch{SO2}. From \citet{Legare05structure}. Reprinted with permission from APS.}
  \label{fig:SO2-232structure}
\end{figure}

While an intuitive way to plot geometries, we will show that such a plot can hide unphysical correlations in the reconstructed geometries. It would be interesting to see the radial distributions of both bond lengths to help ascertain the robustness of their reconstruction method. These marginal distributions are typically of the greatest interest but we will show that they can be also used to hide unphysical correlations (section \ref{ssec:weirdBonds}), and that joint distributions, plotted using a scatter plot for example, should be reported.

To obtain the structures, they assume the explosion system evolves under a purely Coulombic potential and use optimization methods to make guesses at the structure that most accurately reproduces the observed data. Treating the geometry reconstruction as an optimization problem is exactly what we do in chapter \ref{ch:optimization}. However, disappointingly they only allot a couple of sentence to describing their methodology and do not report the optimization methods employed, sidestepping the question of whether their methods were appropriate for the optimization problem as well as the nuances of the reconstruction process that we will discuss in chapters \ref{ch:lookupTable}--\ref{ch:uncertainty}. Without knowledge of the optimization methods used, it is impossible to tell whether appropriate methods were used.

While \citet{Legare05structure} may have not provided sufficient information regarding their methods for a third-party to reproduce their results, they do provide some important insights into the general problem of geometry reconstruction using CEI. They use simulations of dissociative ionization to estimate the role of intermediate charge state dynamics and show that discrepencies between reconstructed geometries and true geometries of the equilibrium state are mainly caused by ion motion during the ionization process. The ion motion is introduced mainly due to the interaction between the molecule and the finite pulse length of the laser (\SI{7}{\femto\s}), which was not as much of a concern with foil-induced dissociation's interaction time of $\sim\SI{0.1}{\fs}$ \citep{Vager89}. They subsequently argue that their half bond length resolution images are sufficient for the observation of large-scale rearrangements of small molecules such as isomerization processes.

% They also claim to have imaged vibrating \ch{D2^+} as well as dissociating \ch{SO2^2+} and \ch{SO2^3+} however they provide no more than a couple of dissociation frames and infer the transient \ch{D2^+} bond length from kinetic energy release ratios as a function of pump-probe time delay \citep{Legare05dynamics}.

\citet{Gagnon08} reported the reconstruction of dichloromethane (\ch{CH2Cl2}) using a home-made\footnotemark~ stochastic-based simulated annealing algorithm that globally optimizes the molecular spatial configuration. Such an algorithm is an example of a heuristic derivative-free optimization algorithm, a class of algorithms that is sometimes described as a ``last resort'', such as by \cite{Conn09} who provide an introduction to derivative-free methods. \citet{Gagnon08} attempt to minimize an objective function in the form of a scaled $\ell_\infty$ norm $\max_i |\mathbf{v}_i - \mathbf{v}_i^\star|/\mathbf{v}_i$ where the index $i$ ranges over the atomic fragments, $\mathbf{v}_i$ is the theoretical or simulated velocity vector for the atomic fragment $i$, and $\mathbf{v}_i^\star$ is the experimentally measured velocity vector for the atomic fragment $i$. \citet[p. 49]{Gagnon06} suggests in his thesis that choosing an objective function in the form of a $\chi^2$ statistics may have been more desirable as it accounts for errors in the velocity vectors as well. Figure \ref{fig:CH2Cl2geometry} shows an example of a reconstruction.

\footnotetext{There is nothing wrong with writing your own code here but nonconvex optimization algorithms are tricky to get right and professional optimization libraries (both proprietary and open-source) do exist. Fortunately, the methodology and implementation's source code are publicly available in the main author's thesis \citep{Gagnon06}.}

\begin{SCfigure}
  \centering
  \includegraphics[width=0.5\textwidth]{gfx/CH2Cl2Geometry}
  \caption
  [An example of a geometry reconstruction of dichloromethane (\ch{CH2Cl2}).]
  {An example of a geometry reconstruction of dichloromethane (\ch{CH2Cl2}) for the \ch{CH2Cl2 $\rightarrow$ H^+ + H^+ + C^+ + Cl^+ + Cl^2+} fragmentation channel. The three-dimensional stochastic motion showing the trajectory of the guesses or iterates of the simulated annealing algorithm is shown (small black dots apparently but looks more like a solid black line) from the initial guesses for the positions (red circles) to the optimal solution found by the algorithm (blue squares). From \citet{Gagnon08}. Reprinted with permission from IOP.}
  \label{fig:CH2Cl2geometry}
\end{SCfigure}

They are only able to obtain the molecular structure for five sets of measured velocity vectors, out of potentially hundreds. Their reconstructed geometries posess the expected tetrahedral structure of \ch{CH2Cl2} and they compare thei reconstructions to spectroscopic measurements, suggesting that the uncertainty in their geometry reconstruction is not due to the algorithm itself, but rather to the uncertainty in the velocity vectors, a relationship we explore quantitatively in chapter \ref{ch:uncertainty}.

The most recent, and perhaps the most interesting geometry reconstruction effort using CEI so far, is the imaging of the long-predicted but experimentally elusive Efimov state of the helium trimer (\ch{4He3}) by \citet{Kunitski15}, coming full circle to the very first images of the hydrogen trimer \citep{Gaillard78}, both excellent examples of non-classical molecular structures which may be imaged effectively by CEI \citep{Vager89}. The ideal Efimov trimer is about $100$ times larger than a typical triatomic molecule and does not exhibit a linear or equilateral structure. An impressive \emph{Physics Today} article by \citet{Greene10} further explores Efimov states and universality in few-body physics. Figure \ref{fig:heliumTrimerReconstruction} shows the theoretical and experimentally reconstructed molecular structures for the helium trimer. They plot their geometries in the center-of-mass coordinate frame which we employ for our geometry plots as well.

\begin{figure}
  \centering
  \includegraphics[width=\textwidth]{gfx/HeliumTrimerReconstruction}
  \caption[Theoretical and experimental molecular geometries of the helium trimer.]
  {Theoretical and experimental molecular geometries of the helium trimer. The (A) theoretical and (B) experimental reconstructed excited-state molecular structures are shown along with the (C) theoretical ground state structure, with a much smaller length scale on both axes. The trimer's center of mass is shifted to the origin. The structures are also rotated such that the principal axis with the smallest moment of inertia points along the $y$-axis, and mirrored with respect to the $x$ or $y$-axis such that one helium atom lies in the first quadrant and the other two in the third and fourth quadrants. From \citet{Kunitski15}. Reprinted with permission from AAAS.}
  \label{fig:heliumTrimerReconstruction}
\end{figure}

\citet[supplementary material]{Kunitski15} describe their reconstruction method which uses a lookup table (with a somewhat similar approach to the one we use in chapter \ref{ch:lookupTable}). The lookup table we describe in chapter \ref{ch:lookupTable} maps geometries $(r_{12}, r_{23}, \theta)$ to asymptotic momentum vectors (described in a dimensionality-reducing convention). However, the lookup table they employed maps geometries $(R_1, R_2, R_3)$ described using Dalitz coordinates to momentum space, also described using Dalitz coordinates. Dalitz coordinates were introduced by \citet{Dalitz53}, and a more modern description of their use in CEI as Dalitz plots (a type of ternary plot) for describing momentum vector arrangements can be found in \citet{Ramadhan16}.

An important feature of Dalitz coordinates is that they require $2$ coordinates to describe a triatomic molecule, effectively reducing the dimensionality of the geometry reconstruction problem to finding the two Dalitz coordinates as opposed to finding the three bond lengths. The third bond length is then calculated from the kinetic energy released (KER) by the three ions. Assuming the Coulomb explosion begins with the molecule at rest, the KER must equal the potential energy of the trimer (in atomic units)
\begin{equation} \label{eq:KER2bonds}
\mathrm{KER} = \frac{1}{R_{12}} + \frac{1}{R_{13}} + \frac{1}{R_{23}}
\end{equation}
and so the kinetic energy release, which is easily calculated from the measured momentum vectors using $\mathrm{KER} = \mathbf{p}^2/2m$, is related to the equalateral structure and a third bond length may be calculated if two others are known. If the ions pick up some initial momentum from the laser ionization process, however, \eqref{eq:KER2bonds} may not hold to a some degree.

They simulate the Coulomb explosion of $1000^2$ structures with different Dalitz coordinates. The forward simulation of the Coulomb explosions is done using Newton's equations of motion while on the other hand we solve Hamilton's equations (section \ref{sec:simulating}). They launch each trajectory six times with different randomly generated small initial momenta whose distribution is taken from measured KER spectra of the single charged helium ion to account for ion motion due to the laser ionization. They find that the initial values of the momenta do not alter their reconstruction results significantly, however initial momentum kicks smaller than one hundredth of the KER ($\sim\SI{0.25}{\eV}$) can produce significant shifts in the Dalitz plot for other molecules such as \ch{OCS^{4+}} in the $(2,1,1)$ fragmentation channel \citep{Ramadhan16}. This introduces additional error when attempting to reconstruct other molecules unless accounted for.

They also notice an ``ambiguity of momentum-to-structure relation'' for a small region in phase space (or structural space) which results in the reconstruction of some irrelevant geometries. This is due to the existence of multiple structures that produce the exact same set of momentum vectors, which we refer to as degenerate geometries in this thesis. We discuss them extensively in the following chapters. It is interesting to note that even a simple structure such the helium trimer's results in degeneracies.

They filter our their degenerate geometries, which seem to be ``irrelevant'' or physically unrealizable as they cannot correspond to the excited state of the helium trimer. While this may be true in this case, it is possible that each degeneracy for another molecular structure may represent a physically realistic geometry (section \ref{sec:optimizationDegeneracies}).

They believe an iterative approach to geometry reconstruction is impracticable due to the highly nonlinear relation between the initial spatial geometry and the asymptotic momentum vectors coupled with the initial momentum each ion picks up during the laser ionization process. While true, especially for the simple structure of the helium trimer, we believe an iterative optimization approach is superior in general. Computing the last bond length from the KER of the system using equation \eqref{eq:KER2bonds} would produce wildly inaccurate geometries as it assumes the process is $100\%$ Coulombic, that is, the potential energy of the molecular system is completely converted into kinetic energy. However, this is not true for many molecules such as \ch{OCS} \citep{Wales14}, and even for highly charged states of some molecules such as \ch{CS2^{10+}} for both the $(4,2,4)$ and $(3,4,3)$ fragmentation channels \citep{Matsuda14}. If we cannot compute the last bond length from the KER of the system then we would have to start with a three-dimensional lookup table, and lookup table storage space requirements scale up in an exponential manner (section \ref{ssec:LTspace}). 

They suggest that extending their approach to four-body system seems feasible, in which case four-particle Dalitz plots may be used \citep{Schulz07}, and may be an idea worth pursuing for cases when \eqref{eq:KER2bonds} holds.

% \subsubsection*{Inferring dynamics from energy spectra and momentum vector distribtuions}
% Due to the difficulty of recovering the molecular geometries, the structures and dynamics of small molecules are usually inferred from the kinetic energy spectra of the fragments and their momentum vector distributions (usually studied using Newton and Dalitz plots).

% \subsection{Other methods and uses}

% \subsection{Molecular movies}
% Molecular movies are of course not only of interest in physics and chemistry as a means of probing fundamental processes, but also in the biological sciences where molecular structure play a crucial role in determining the function of biomolecules such as proteins. However, the molecules of interest there are much too large to be studied by any of the previous techniques. Thus molecular movies in the biological sciences tend to be annotated computer simulations amalgamated from multiple studies. That said, they are very impressive pieces of work.

% A particularly impressive movie by \citet{Cheung12} showcases the process of RNA polymerase transcription and goes on for over six minutes.