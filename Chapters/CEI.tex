\chapter{Coulomb explosion imaging}\label{ch:CEI}

\vspace{-1.5 em}
\minitoc\hrule
\vspace{1.5 em}

\section{Physical principles}

\section{Achievements}
\subsection{Foil-forged images}
\subsection{CEI by highly charged ion impact}
\subsection{Laser-induced CEI}

\section{Outline of the experiment}

\subsection{Pump-probe Coulomb explosion imaging}
In pump-probe Coulomb explosion imaging (CEI) one ultrashort laser pulse is split into two pulses through the use an asymmetric beamsplitter. One of the pulses, the pump pulse, is usually much weaker than the other, the probe pulse. A time delay $\tau$ between the pulses is then created such that the pump pulse goes first and the probe pulse second. The job of the pump pulse will be to initiate some change in the molecule. One example could include an isomerization of the molecule. Thus the pump pulse ``pumps'' the molecule into some excited state. The job of the powerful probe pulse is to engulf the molecule in an intense enough laser field such that multiple electrons are stripped off of it. The molecule's individual atoms are left in a highly-charged state and begin to behave as individual point charges in a purely Coulombic potential. The entire process occurs in the presence of a constant electric field and so the positively-charged ions all accelerate upwards towards a time- and position-sensitive detector. Thus the probe pulse allows for the ``probing'' of the excited state.

\subsection{Femtosecond Multiple Pulse Length Spectroscopy}

\section{Experimental apparatus}

\section{Data analysis}

\subsection{Time and position measurement}
The position is then calculated using
\begin{equation}
x = \frac{Q_1 + Q_2}{Q_1 + Q_2 + Q_3 + Q_4} ,\quad
y = \frac{Q_1 + Q_3}{Q_1 + Q_2 + Q_3 + Q_4}
\end{equation}

\subsection{Calculating the atomic fragments' momenta}
The components of the three-dimensional momentum vector $\mathbf{p} = (p_x,p_y,p_z)$ for each atom are then calculated as

\begin{equation}\label{eq:CEImomenta}
p_x = \frac{m(x-x_0)}{t} ,\;
p_y = \frac{m(y-y_0)}{t} ,\;
p_z = \frac{qV}{2\ell} \left( \frac{t_0^2 - t^2}{t} \right)
\end{equation}
where $m$ is the atom's mass, $(x,y)$ is the location the atom collided with the MCP detector, and $(x_0,y_0)$ is the location that the Coulomb explosion originated. The location $(0,0)$ corresponds to the physical center of the MCP detector. $q$ is the net charge of the atom, $V$ is the value of constant electric potential the atom is subjected to, and $\ell$ is the distance from the location of the Coulomb explosion to the detector. $t$ is measured time of flight (between Coulomb explosion and detection) of the atom and 
\begin{equation}
t_0 = \sqrt{\frac{2d\ell}{V} \left( \frac{m}{q} \right)}
\end{equation}
is the atom's time of flight assuming no external forces act on it during its trip to the detector.

\subsection{Measurement uncertainty in the momenta}
For any relation $f = f(x_1, x_2, \dots, x_n)$, assuming independent variables, the absolute uncertainty in $f$, denoted $\Delta f$, may be calculated as
\begin{equation}
\Delta f = \sqrt{\sum_{i=1}^{n} \left( \frac{\partial f}{\partial x_i} \Delta x_i \right)^2}
\end{equation}
where $\Delta x_i$ is the uncertainty in the independent variable $x_i$.

Using this we may calculate the uncertainty in the measured momentum values, which will we different for each component. In our case, $p_x = p_x(m,x,x_0,t)$ and $p_y = p_y(m,y,y_0,t)$, however, the uncertainty in the atomic mass $m$ is orders of magnitude smaller than the uncertainty in the other variables and so we will ignore its effects. Thus we get that
\begin{subequations}
  \begin{align}
  (\Delta p_x)^2 &= \frac{\partial p_x}{\partial x}\Delta x + \frac{\partial p_x}{\partial x_0}\Delta x_0 + \frac{\partial p_x}{\partial t}\Delta t \\
  (\Delta p_y)^2 &= \frac{\partial p_y}{\partial y}\Delta y + \frac{\partial p_y}{\partial y_0}\Delta y_0 + \frac{\partial p_y}{\partial t}\Delta t
  \end{align}
\end{subequations}
where the partial derivatives can be calculated from \eqref{eq:CEImomenta} as
\begin{subequations}
  \begin{align}
  \frac{\partial p_x}{\partial x} = \frac{m}{t} &,\quad \frac{\partial p_x}{\partial x_0} = \frac{m}{t} ,\quad \frac{\partial p_x}{\partial t} = -m\frac{x-x_0}{t^2}\\
  \frac{\partial p_y}{\partial y} = \frac{m}{t} &,\quad \frac{\partial p_y}{\partial y_0} = \frac{m}{t} ,\quad \frac{\partial p_y}{\partial t} = -m\frac{y-y_0}{t^2}
  \end{align}
\end{subequations}
and so
\begin{subequations}
  \begin{align}
  \left( \frac{\Delta p_x}{p_x} \right)^2 &= \left( \frac{\Delta x}{x - x_0} \right)^2 + \left( \frac{\Delta x_0}{x - x_0} \right)^2 + \left( \frac{\Delta t}{t} \right)^2 \\
  \left( \frac{\Delta p_y}{p_y} \right)^2 &= \left( \frac{\Delta y}{y - y_0} \right)^2 + \left( \frac{\Delta y_0}{y - y_0} \right)^2 + \left( \frac{\Delta t}{t} \right)^2
  \end{align}
\end{subequations}

Repeating the process for $p_z = p_z(q,V,\ell,t_0,t)$ but ignoring the tiny uncertainties in $q$, $V$, and $\ell$, we get

\section{Computationally simulating a Coulomb explosion} \label{sec:simulating}
To simulate an explosion of a molecule containing $n$ atoms, we must solve the classical equations of motion for each ion right after the explosion. We choose to use Hamiltonian mechanics here to acquire a system of first-order differential equations which may be easily solved by numerical methods such as the ubiquitous fourth-order Runge-Kutta. Assuming a purely electromagnetic potential for each ion, the Hamiltonian of the molecular system is
\begin{equation}
\mathcal{H}(\mathbf{r}_i, \mathbf{p}_i, t) = \sum_{i=1}^n \frac{\mathbf{p}_i^2}{2m_i} + \frac{1}{4\pi\epsilon_0}\sum_{\substack{\lbrace i,j\rbrace\\ i \ne j}} \frac{q_iq_j}{|\mathbf{r}_i-\mathbf{r}_j|}
\end{equation}
where $i,j \in \lbrace 1,2,\dots, n \rbrace$ and so the second summation is over all $i,j$ pairs where $i \ne j$. Calculating Hamilton's equations for the system, we get
\begin{subequations}
  \begin{align}
  \frac{d\mathbf{r}_i}{dt} &= \frac{\partial \mathcal{H}}{\partial \mathbf{p}_i} = \frac{\mathbf{p}_i}{m_i} \\
  \frac{d\mathbf{p}_i}{dt} &= \frac{\partial \mathcal{H}}{\partial \mathbf{r}_i} = \frac{1}{4\pi\epsilon_0}\sum_{j, \; j \ne i} \frac{\mathbf{r}_i - \mathbf{r}_j}{|\mathbf{r}_i - \mathbf{r}_j|^3}
  \end{align}
\end{subequations}
where $i$ is held fixed over the second summation. With appropriate initial conditions this system of $6n$ scalar first-order ordinary differential equations may be easily solved using, for example, the classical fourth-order Runge-Kutta method for numerically solving ordinary differential equations. The atoms are assumed to be at rest so that $\mathbf{p}_i(t=0) = 0$, while the initial positions, $\mathbf{r}_i(t=0) = 0$, are chosen to correspond to the molecular geometry. \marginpar{Discuss the validity of the at rest assumption.}

One way to think of the problem being tackled in this thesis is: which initial geometry $\mathbf{r}_i(t=0) = 0$ results in the momentum values measured at the detector? The atoms are far enough apart after just a few nanoseconds that by the time they arrive at the detector, they feel almost no forces due to each other and their momenta attain asymptotic values which we can denote $\mathbf{p}_i(t\rightarrow\infty)$.

\section{Conventions for geometries and momenta}

\subsection{Describing molecular geometries by a Z-matrix}

\subsection{A homemade convention to describe momentum vectors}
