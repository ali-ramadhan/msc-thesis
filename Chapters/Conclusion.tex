\chapter{Conclusion}\label{ch:conclusion}

%\vspace{-1.5 em}
%\begin{addmargin}[-0.5cm]{0cm}
%  \minitoc
%\end{addmargin}
%\hrule
%\vspace{1.5 em}

We have shown that geometry reconstruction can be performed quickly using two approaches, a heuristic lookup table discussed in chapter \ref{ch:lookupTable} and the more precise optimization routine relying on constrained nonlinear optimization methods in chapter \ref{ch:optimization}. We used both methods to investigate the existence of multiple solutions, or degenerate geometries, and used different analyses to study their nature which should be performed before attempting to reconstruct any molecular geometry. We also investigated the effect of uncertainty in the momentum vectors on the reconstructed geometries in chapter \ref{ch:uncertainty}, highlighting the sensitive nature of the task.

\section{Infeasibility of geometry reconstruction}
We have found two major barriers to accurate geometry reconstruction. Firstly, it is highly sensitive to uncertainties in the momentum vectors, however, even if they are measured accurately enough that measurement uncertainty is no longer a concern, the initial momentum carried by each atomic fragment would need to be measured accurately as well since they introduce additional uncertainty that cannot be inferred otherwise. Secondly, the existence of degenerate geometries for large regions of phase space may make reconstruction difficult, especially if multiple degeneracies represent physically realizable structures. The problem is exasperated in the presence of uncertainty as the true geometry may now be contained in multiple unconnected regions in phase space.

We only investigated the simplest case possible by assuming the atomic fragments evolve on a purely classical Coulomb potential and that they possess zero initial momentum, and yet the reconstruction problem already seems to exhibit pathological behavior. Accounting for additional complexity through the use of a more accurate non-Coulombic potential, the measurement or estimation of initial momentum for the atomic fragments, and modeling the effect of the laser's electric field with the molecule are likely to increase the problem's sensitivity to uncertainty in the momentum vectors, and thus the entire task of geometry reconstruction appears to be quite complex and only possible under certain conditions, and becomes even more complex for larger molecules which require at least twice the number of degrees of freedom to describe. Additional degrees of freedom, introduced due to the modeling of additional phenomena or the study of larger molecules, would increase the dimensionality of the optimization problem, possibly resulting in a problem that may be difficult to solve with the already advanced optimization methods employed due to the curse of dimensionality. It is also possible that accounting for additional complexity may introduce additional degeneracies through the introduction of extraneous degrees of freedom, which would make the accurate reconstruction of molecular geometries even more difficult.

\section{A framework for geometry reconstruction using CEI}
While geometry reconstruction using CEI can be difficult, it may be performed under certain conditions:
\begin{enumerate}
  \item The momentum vectors are measured very accurately, with an error less than $1\%$.
  \item The atomic fragments carry very little initial momentum. Or in the case that they do not, then the initial momenta of the atomic fragments must be measured experimentally, or estimated accurately (which would add further complexity to the geometry reconstruction problem).
  \item Negligible molecular rearrangement occurs during the ionization process. This would require at least, the use of few-cycle laser pulses and depends on the molecule as well, as some may have a tendency to rearrange significantly on a time scale shorter than the shortest intense laser pulses.
  \item The reconstructed geometries are not degenerate with other geometries. Thus regions of phase space containing degenerate geomtries must be mapped out.
  \item The Coulomb potential very closely approximates the potential experienced by the atomic fragments during the Coulomb explosion.
\end{enumerate}

If these conditions above hold, then we propose the following framework for geometry reconstruction:
\begin{enumerate}
  \item Choose conventions for describing the molecule's structure and the orientation of the asymptotic momentum vectors following a Coulomb explosion, for example, as described in section \ref{sec:conventions}..
  \item Map out the degenerate regions of your molecular system, that is, find the regions of phase space which contain degenerate geometries (as done in section \ref{sec:optimizationDegeneracies}). You may choose to completely ignore geometries from these regions if multiple degeneracies correspond to physically realizable geometries.
  \item Filter out momentum vector measurements that exhibit a large error, such as by not summing to zero, or that correspond to an explosion that is highly non-Coulombic, that is the kinetic energy of the atomic fragments significantly differs from the kinetic energy expected by a purely Coulombic explosion. If not using the Coulomb potential, then compare with simulations using the more appropriate potential.
  \item Set up the optimization problem (section \ref{sec:optImplementation}) or the statistical model (section \ref{sec:uncertaintyBayesian}) for geometry reconstruction. Inequality constraints may be used to constrain the reconstructed geometries to have feasible parameters. If using Bayesian inference is used, quantum chemistry simulations of the ground or excited state may provide prior distributions, otherwise uninformative priors may be used.
  \item If using Bayesian inference, then the calculated posterior distributions should already have taken into account the effect of uncertainty. If not, the uncertainty in the reconstructed geometries must be quantified.
  \item Plot the final geometry. Bond length and bond angle (as well as dihedral angles) distributions should be reported, as well as the correlations between each of these parameters (section \ref{ssec:weirdBonds}). This may constitute a single frame of a molecular movie.
  \item Repeat this process for each frame, which may be for each pulse length in a pump-probe CEI experiment, thus producing a molecular movie.
\end{enumerate}

% \section*{Possible improvements}
% \section*{Future directions}